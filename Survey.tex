\documentclass[a4paper]{article}
\usepackage{fullpage}

\title{Robotic Swarms: \\Distributed Coordination Without Location}
\author{S.J.A. Bekhoven  \and
    S.P. Metman \and
    M.J. Rogalla}
\date{\today}

\pagestyle{empty}

\begin{document}
\maketitle
\thispagestyle{empty}

\begin{abstract}
This is the abstract of my paper.
This is the abstract of my paper.
This is the abstract of my paper.
This is the abstract of my paper.
This is the abstract of my paper.
This is the abstract of my paper.
This is the abstract of my paper.
This is the abstract of my paper.
This is the abstract of my paper.
This is the abstract of my paper.
This is the abstract of my paper.
This is the abstract of my paper.
\end{abstract}


\section{Introduction}
  In order to emphasize the importance of the connection between the technology and its applications, a top-down approach is used for this survey. For each of the applications, an analysis is made of all the required services.

  \subsection{Region Covering}
  \subsection{Dangers}
  \subsubsection{Swarm-Assisted Fire Fighting}
  \textbf{Description: }\emph{Swarm-Assisted Fire Fighting makes interactive use of autonomous robots in fire emergency settings. These swarms of robots are capable of supporting and enhancing fire fighting operations co-operatively with each other and are coordinated by a single human supervisor.}\cite{Naghsh2008,Penders2011}\\\\
  The services required for Swarm-Assisted Fire Fighting include, but are not limited to: \emph{foraging}, \emph{dispersion}, \emph{mapping} and \emph{exploration}.\cite{Naghsh2008,Penders2011} The foraging services are needed in order to give the swarm the ability to search and locate victims. Dispersion is required in order for the swarm to navigate optimally and have non-conflicting exploration. The mapping and exploration service are required to create a well constructed map of the explored area, such that the human and other robots are aware of their surroundings, even if it is impossible to get a visual due to reduced visibility caused by smoke. 
  \subsection{Swarm Radiation Source Discovery}
  \subsection{Scaling in time}
  \subsection{Redundancy}
  
  \emph{Lorem ipsum\ldots}


  \subsection{\ldots}
  Give an overview of real-world applications possible with Robotic Swarms. A list of possible applications:
  \begin{enumerate}
    \item Cleaning
    \item Space Exploration (swarm of Mars rovers)
    \item Rescue Missions
    \item Treacherous Radioactive Survey
    \item Survey and cleanup of Toxic Spills
    \item Surveillance
  \end{enumerate}

\section{Definitions from Literature}

  \subsection{Orientation}
  Location-based vs Range-based \ldots
  \subsection{Applications}
  List of applications.... idea for table: Orientation Table: LB RB LF RF\\
  
  \begin{tabular}{|p{3cm}|p{3cm}|p{3cm}|}
    \hline
     & Location-based & Location-free\\\hline
    Range-based & \ldots & \\\hline
    Range-free & \ldots & \\
    \hline
  \end{tabular}
  
  \subsubsection{Service Required}

\section{In-depth review of Services}
    \ldots small introduction
    \subsection{Service 1}

      \ldots Introduction to problem 1.

      \subsubsection{Comparison of Solutions}
      \subsubsection{Remaining Problems}

    \subsection{Service 2}

      \ldots Introduction to problem 2.

      \subsubsection{Comparison of Solutions}
      \subsubsection{Remaining Problems}

    \subsection{Service 3}
      \ldots Introduction to problem 3.

      \subsubsection{Comparison of Solutions}
      \subsubsection{Remaining Problems}

    \subsection{Service 4}
      \ldots Introduction to problem 4.

      \subsubsection{Comparison of Solutions}
      \subsubsection{Remaining Problems}

    \subsection{Overview}
      \ldots Overview of all Applications.
      
      \begin{tabular}{|p{3cm}|p{3cm}|p{3cm}|p{3cm}|}
        \hline
        % after \\: \hline or \cline{col1-col2} \cline{col3-col4} ...
        Application & Services & Algorithms & Problems \\\hline
        \ldots & \ldots & \ldots & \ldots \\\hline
        \ldots & \ldots & \ldots & \ldots \\\hline
        \ldots & \ldots & \ldots & \ldots \\
        \hline
      \end{tabular}

\section{Unsolved Problems}

  \ldots Summarize remaining problems.

\section{Discussion}

  Make references in the running text with the \verb+\cite+
  command \cite{dijkstra68}. Multiple referrences go like this
  \cite{charniak85,steels98}.


  


\bibliographystyle{IEEEtran}
\bibliography{Bibliography}

\end{document}








