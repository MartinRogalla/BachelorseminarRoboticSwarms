%!TEX root = ../../Bachelorseminar-RoboticSwarms.tex

Path-finding is the problem of a robotic swarm for each robot to find the optimal collistion-free path from the start location to the target location. \cite{qin2004path}.
Another term for path-finding is path-planning. 
This is a problem found in many robotic swarms applications.
This is because when robots move to a location in any distributed algorithm, this should be done as fast as possible. 
Path finding can be used in multiple real-life applications.
These applications include trying to find the optimal path to food sources and bringing food back to the base, which is often called foraging \cite{hoff2010two}.
Another application is finding an optimal path to survivors from a disaster, for search-and-rescue operations.
Of course, this can also be extended to path-finding for oil spills, crime victims or other targets.\cite{pugh2007inspiring} 
Path-finding algorithms are often combined with some form of particle swarm optimization. \cite{poli2007particle}. \\

There are many algorithms that can be used to solve the path-finding problem. 
We will point out a few of the most important and recent algorithms. \cite{hoff2010two} \cite{Bhattacharjee} \cite{Jung} \cite{Kala}
We will discuss these algorithms, pointing out how they work and what problems are solved. \\

The first algorithm is the foraging algorithm. \cite{hoff2010two}
Foraging algorithms are based on the foraging behavior of ants to find food sources. 
Ants can leave chemical pheromones behind for other ants, which they use to decide which way they go. 
Two different algorithms can be used for this: the Virtual Pheromone algorithm and the Cardinality algorithm.
The Virtual Pheromone algorithm uses two different types marks instead of many, to keep the robots low-level. 
For the same reason, the marks are made virtual by direct local communication between robots in the swarm.
The core of the algorithm is that some robots will stop their search for the target food source, and become 'pheromone robots', also known as beacons.
These beacons send out marks as floating point numbersfor other foraging swarm robots, telling them where or where not to move. \\
The second foraging algorithm is the Cardinality algorithm. 
This works similar to the Virtual Pheromone algorithm, as in that walker robots can become beacons. 
The difference is that the beacons send out integer values, indicating how many other beacons it would hear, including the nest beacon. 
Every integer value of beacon can then be interpreted as the number of beacons between that beacon and the nest. \\
Conclusive results are produced in the article that the Cardinality-algorithm outperforms the Virtual Pheromone algorithm, proving that the ant pheromone algorithm can be improved.
These two algorithms both use only local communication and every robot does not detect exact range between other robots.
So, the two algorithms are bot location-free and range-free. \\

The second algorithm we'll discuss is the Artificial Bee Colony algorithm. \cite{Bhattacharjee}
Again, this algorithm is inspired by nature, but this time is based on the stoach behavior of foraging in bees.
In this algorithm, the robots (from this point on referenced as bees), can be divided in three groups:
\begin{itemize}
	\item A bee waiting on a dance area for making decision to choose a food source is called an onlooker.
	\item A bee going to the food source visited by it previously is named as employed bee.
	\item A bee carrying out random search is called a scout.
\end{itemize}
The algorithm then consists of four phases:
\begin{enumerate}
	\item The behaviors are randomly distributed among all the robots (according to a certain distribution).
	\item The employed bees move to find a new or better food source, if they can find it based on visual information. 
	\item The onlooker bees then calculate the best places to gather food depending on the input from the employed bees.
	\item Finally, if there are food sources of which the path to it can not be improved and have reached a limit, they are abandoned and replaced by scouts.
\end{enumerate}

Steps 2, 3, and 4 are continuously repeated, until an optimal path is created for all the employed bees.
This algorithm does make use of a given environment and the path-finding is done with local sensors. 
So, this algorithm is location-based and range-based.
