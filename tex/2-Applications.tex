%!TEX root = ../Bachelorseminar-RoboticSwarms.tex
Robotic swarms can be used for many real-world applications as for example in tasks that cover a region, tasks that are to dangerous for human beings, tasks that scale-up or scale-down in time or tasks that require redundancy \cite{csahin2005swarm}. In this chapter, we will name a few of the most-used techniques used by actual applications of robotic swarms, and give examples of applications that use these techniques.

  \subsection{Exploration}
  %!TEX root = ../Bachelorseminar-RoboticSwarms.tex

Exploring an environment is one of the fundamental problems faced in mobile robotics. The main goal is to minimize the overall exploration time and the main problem faced when trying to achieve this goal is finding appropriate target points for each individual robot so that they simultaneously explore different regions of the area \cite{burgard2005coordinated}. Robotic swarm exploration can be used for real-world applications like rescue missions \cite{Naghsh2008,Penders2011}, surveillance \cite{Burkle2010} and cleaning \cite{wagner2008cooperative}.
  
  \subsection{Mapping}
  %!TEX root = ../../Bachelorseminar-RoboticSwarms.tex

Mapping is a technique in which a swarm of robots try to make a map of the surrounding unknown environment. 
The mapping of an unkown environment can be efficiently done with the usage of robotic swarms; this is called collaborative mapping. 
As with other algorithms, this mapping should be done as fast as possible without losing accuracy. 
The core of the mapping technique looks very similar to exploration, but an important difference is that in mapping a map should be made and stored. 
This leads to more communication between robots and more distributed intelligence. \\

Mapping of an unknown environment is very useful in hazardous and unaccessable environments. \cite{hardin2004modified}
In this paper a practical application is mentioned, the detection of hazardous aerosol in a contaminated, confined area. 
Robotic swarms are used in this work because it is safer to use them. 
Of course, as mentioned earlier, some techniques overlap, and mapping is no exception to this. 
The mapping technique uses elements of exploration, dispersion and localization.  

%OLD
%The mapping of an unkown environment can be conveniently done with the usage of robotic swarms, this is called collaborative mapping. 
%Collaborative mapping is very useful in hazourous and unaccessable environments.\cite{hardin2004modified} 
In \cite{sheng2006distributed} a model is described in which the robots have limited communication range and full knowledge of their location. By continuously looking for the closest undiscovered area in combination with a nearest measure it tends to explore the complete area by staying together and continiously updating each others maps.
Some of the techniques used in these types of applications are localization, exploration, dispersion, mapping. \cite{sheng2006distributed,rothermich2005distributed} 
  
  \subsection{Dispersion}
 %!TEX root = ../../Bachelorseminar-RoboticSwarms.tex

  \emph{Dispersive applications make interactive use of autonomous robots in exploratory settings. The Robotic-Swarms are capable of supporting and enhancing operations co-operatively and are optionally coordinated by a single human supervisor. Possible real world applications include: military response, disaster response and planetary exploration}\cite{ludwig2006robotic,Penders2011}
    The techniques required for these types of dispersion include, but are not limited to: \emph{foraging}, \emph{dispersion}, \emph{mapping}  and \emph{exploration}.\cite{morlok2007dispersing,Penders2011,mclurkin2007distributed} The foraging techniques are needed in order to give the swarm the ability to search and locate subjects. Dispersion is required in order for the swarm to navigate optimally and prevent conflicts in exploration. The mapping and exploration techniques are required to create a well constructed map of the explored areea, such that the human and other robots are aware of their surroundings, even if it is impossible to get a visual due to reduced visibility.\cite{morlok2007dispersing}
    

  \subsection{Localization}
 %!TEX root = ../Bachelorseminar-RoboticSwarms.tex

  \emph{Due to the ambiguity of this name, there have been many applications which have aimed to achieve this, from nuclear spills and oil spills to fire-origins. Much of this however is theoretical work, due to the fact that the price of these individual robots is still rather high.}

  Some of the techniques used in toxic source discovery are: control, communication and distribution.\cite{Li2012}

 \subsection{Path-finding}
 %!TEX root = ../Bachelorseminar-RoboticSwarms.tex

  Path finding or path planning (in this paper we will use path finding) basically is the basis of every robotic swarm technique, since every distributed algorithm in mobile robotics tries to find the best location to go to from its current location. In this paper we consider path finding as the problem to find the optimal collision free path from the start state to the goal state. \cite{qin2004path}. Path finding can be used in multiple real-life applications such as foraging (trying to find food sources and bring food back to the basis) \cite{hoff2010two} and searching (for spills, victims, targets) \cite{pugh2007inspiring} and is often combined with some form of particle swarm optimization \cite{poli2007particle}.

\subsection{Collective transport}
%!TEX root = ../../Bachelorseminar-RoboticSwarms.tex

Collective transport of objects is the technique of a swarm of robots locating an object and collectively moving the object to another place, like a homebase. This can be compared to a foraging technique; although this implies that a path is made to a certain place, which does not apply to all collective transport techniques. \cite{hoff2010two}  \\
Transporting objects by robotic swarms has many potential applications in many settings, from agriculture to construction to disaster relief. Especially in dangerous settings like warzones or radio-active areas, robotic swarms can be a powerful tool to safely retrieve many objects. Especially because the robots used for this application are cheap to produce. \\
For this application to work, each agent only has to know the target direction but does not have to know the object shape, weight, its own position or the position and number of other agents. This makes this application location-free and range-free. An extra invariant is that there must be enough robots to overcome the static friction of the object to be transported.  \cite{Rubenstein}


\subsection{Surveillance}
%!TEX root = ../../Bachelorseminar-RoboticSwarms.tex

Surveillance is a technique in which a swarm of robots continuously patrol the same space, and simultaneously keep an eye out for abnormalities which are specified beforehand. Often this technique also includes formation, because patrolling robots should be in the same formation. Furthermore, in unknown locations mapping would be used, and in locating abnormalities a localization technique is often used. \\
Surveillance with a single robot is already a powerful tool, but with a swarm of robots it becomes even more effective. A larger area can be covered, local communication can be handled efficiently and depleted robots can interchanged with other robots. The applications of swarm surveillance are diverse, and are already used in a vast range of applications like agricultural practices, police surveillance, inspecting unreachable locations, patrol missions and reconnaisance tasks. \cite{Burkle2010} \\
The application can be implemented in many different ways, but is able to work location-free and range-free, but in some applications the location is already known and in others it can be helpful for a robot to know the exact location of the other robots of the swarm. In some applications the process isn't even completely autonomous; a human user could be in a ground control station. Some of the techniques used in swarm surveillance are: dispersion, localization, distributed communication and exploration. 

\subsection{Formation}
%!TEX root = ../Bachelorseminar-RoboticSwarms.tex

  \subsection{Exploration}
  \textbf{Description: }\emph{Cleaning/exploring a component in an unkown area is a common problem which can be solved effectively by using a tobotic swarm. For example in \cite{wagner2008cooperative} a swarm of robots is placed in a certain component which has to be cleaned totally. The robots have limited visibility and can only see other robots within a certain distance. Main goal is to not disconnect the component that has to be cleaned to be sure that the component will be get fully cleaned in the end.} \\
  The main techniques that are described in \cite{wagner2008cooperative} are algorithms that preserve the connectivity of the spot to be cleaned and prevent the clustering of robots.




  
  \subsection{Overview}
  Give an overview of real-world applications possible with Robotic Swarms. A list of possible applications:
    \begin{enumerate}
      \item Area Cleaning
      \item Space Exploration (swarm of Mars rovers)
      \item Rescue Missions
      \item Treacherous Radioactive Survey
      \item Survey and cleanup of Toxic Spills
      \item Surveillance
      \item Swarm-Assisted Fire Fighting
     \item Collective Transport of Complex Objects
    \end{enumerate}
  \textbf{Categories}
    \begin{itemize}
      \item Region Covering
      \item Dangers
      \item Scaling in time
      \item Redundancy
    \end{itemize}
  