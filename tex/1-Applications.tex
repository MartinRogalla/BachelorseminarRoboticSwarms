%!TEX root = ../Bachelorseminar-RoboticSwarms.tex
Robotic swarms can be used for many real-world applications as for example in tasks that cover a region, tasks that are to dangerous for human beings, tasks that scale-up or scale-down in time or tasks that require redundancy \cite{csahin2005swarm}.

\subsection{Cleaning}
  \textbf{Description: }\emph{Cleaning a certain area is a common problem which can be solved effectively by using a swarm of autonomous robots. These robots are used to fully explore and clean the area while only being able to see their neighbourhood. In case of \cite{altshuler2005swarm} robots clean areas without disconnecting the dirty area, so that, in case of failing robots, the area will still be cleaned completely.} \cite{altshuler2005swarm,wagner2008cooperative}
  
  \subsection{Swarm-Assisted Fire Fighting}
  \textbf{Description: }\emph{Swarm-Assisted Fire Fighting makes interactive use of autonomous robots in fire emergency settings. These swarms of robots are capable of supporting and enhancing fire fighting operations co-operatively with each other and are coordinated by a single human supervisor.}\cite{Naghsh2008,Penders2011}\\

  The techniques required for Swarm-Assisted Fire Fighting include, but are not limited to: \emph{foraging}, \emph{formation}, \emph{mapping} and \emph{exploration}.\cite{Naghsh2008,Penders2011}\todo[inline]{Check more sources and verify techniques} The foraging techniques are needed in order to give the swarm the ability to search and locate victims. Formation is required in order for the swarm to navigate optimally and prevent conflicts in exploration. The mapping and exploration service are required to create a well constructed map of the explored area, such that the human and other robots are aware of their surroundings, even if it is impossible to get a visual due to reduced visibility caused by smoke.

  \subsection{Toxic Emition Source Discovery by Robotic-Swarms}
  \textbf{Description: }\emph{Due to the ambiguity of this name, there have been many applications which have aimed to achieve this, from nuclear spills and oil spills to fire-origins. Much of this however is theoretical work, due to the fact that the price of these individual robots is still rather high.}

  Some of the techniques used in toxic source discovery are: control, communication and distribution.\cite{Li2012}
  
  \subsection{Draft}
  Give an overview of real-world applications possible with Robotic Swarms. A list of possible applications:
    \begin{enumerate}
      \item Cleaning
      \item Space Exploration (swarm of Mars rovers)
      \item Rescue Missions
      \item Treacherous Radioactive Survey
      \item Survey and cleanup of Toxic Spills
      \item Surveillance
    \end{enumerate}
  \textbf{Categories}
    \begin{itemize}
      \item Region Covering
      \item Dangers
      \item Scaling in time
      \item Redundancy
    \end{itemize}
  