%!TEX root = ../../Bachelorseminar-RoboticSwarms.tex

Dispersion is a technique in which a swarm of robots try to cover an area as large as possible.  
Dispersive applications make interactive use of autonomous robots in exploratory settings, and is thus often used in combination with exploration techniques. 
The robotic swarms are capable of supporting and enhancing operations co-operatively and are optionally coordinated by a single human supervisor. 
There is one big difference between dispersion and exploration, although at first they seem very much alike. 
The difference is that in exploration (and mapping) the whole area has to be explored, but in dispersion an area as large as possible should be covered. 
This means that in dispersion, the ultimate goal is to cover the whole area as fast as possible, but the robotic swarm has a possibility to stop exploring when reaching its maximum coverage. \\

An example of an application is, among others, planetary exploration. \cite{ludwig2006robotic,Penders2011}
In this example, a swarm of robots tries to explore a part of the planet, but has no intent to explore the whole planet. 
So, a dedicated dispersion algorithm is used. 
Other examples of applications include military response and disaster response applications. 

%OLD
%Dispersive applications make interactive use of autonomous robots in exploratory settings. 
%The robotic swarms are capable of supporting and enhancing operations co-operatively and are optionally coordinated by a single human supervisor. 
%Possible real world applications include: military response, disaster response and planetary exploration\cite{ludwig2006robotic,Penders2011}
%The techniques required for these types of dispersion include, but are not limited to: \emph{foraging}, \emph{dispersion}, \emph{mapping}  and \emph{exploration}.\cite{morlok2007dispersing,Penders2011,mclurkin2007distributed} The foraging techniques are needed in order to give the swarm the ability to search and locate subjects. Dispersion is required in order for the swarm to navigate optimally and prevent conflicts in exploration. The mapping and exploration techniques are required to create a well constructed map of the explored areea, such that the human and other robots are aware of their surroundings, even if it is impossible to get a visual due to reduced visibility.\cite{morlok2007dispersing}
    
%!TEX root = ../../Bachelorseminar-RoboticSwarms.tex

One of the key techniques used in robotic swarms is dispersion. The goal of dispersion in the context of robotic swarms is to scatter the individual robots in an environment such that every section of the environment can be covered. In this section we will focus on non-location oriented algorithms. If prior knowledge about the exact location is provided, a formation technique should be used.

\subsubsection{Algorithms}
  The following dispersion algorithms are the most widely used algorithms and thus will be taken into our analysis. A very brief description is given of each algorithm. For more information on the operation of these algorithms, please consult the references.\\

  \textbf{Location-Free Dispersion Algorithms}
  \begin{itemize}
    \item \textbf{Depth-First Leader-Follower Strategy (\emph{DFLF})}\cite{hsiang2004algorithms}:\\
      A \emph{Depth First Search}(DFS) inspired algorithm in which the swarm has one leader at any given point in time. The robotic swarm has the overview of a map in which specific regions, called \emph{pixels}, are defined. \emph{Frontier pixels}, are pixels which haven't been traversed yet. The leader robot keeps looking for frontier pixels until there are none left and stops and tells its \emph{successor robot} to be the leader. If there are no successors left, the algorithm halts and the total dispersion of the area. The other robots always try to follow the leader and traverse the tiles around it.
    \item \textbf{Breadth-First Leader-Follower (\emph{BFLF})}\cite{hsiang2004algorithms}:\\
      A \emph{Breadth First Search}(\emph{BFS}) inspired algorithm, which does not exactly perform \emph{BFS}, but approximates it. It is a more complex algorithm than the \emph{DFLF} algorithm, but has to make fewer moves to fully cover the map. In an extension of the \emph{DFLF} algorithm, the \emph{BFLF} algorithm also contains a waiting state, in which a robot pauses to make the next move. Furthermore, instead of having only one leader, this algorithm allows for multiple leaders. The leaders again strive to find all the frontier pixels, but now also make sure that the follower robots don't stray to far away. Once no frontier pixels can be found by the leader, the leader waits for the followers to arrive and passes on its leadership to one of the followers, the successor. The \emph{BFLF} strategy tries to create as many paths as possible. Visited pixels form a tree, the tree can be branched, which then represent the alternate pixels reachable from that pixel. This branching balances the flow by giving the possiblity to go through pixels multiple times, to be able to go into differerent directions.
  \end{itemize}
  \textbf{Range-Based Dispersion Algorithms}\\
  \begin{itemize}
    \item \textbf{Directed Dispersion/Disperse Uniformly}\cite{mclurkin2007distributed}:\\
      The directed dispersion algorithm has the goal to disperse the robots quickly and uniformly, while keeping the robots connected. The algorithm consists of two sub-algorithms: \emph{disperseUniformly} and \emph{frontierGuidedDispersion}. The \emph{disperseUniformly} algorithm spreads the robots evenly with given constraints. It works by calculating an opposite direction vector of the positions of the nearest robots. This means that communication between robots is of vital importance. The \emph{frontierGuidedDispersion} directs robots towards unexplored areas, while keeping the robots connected. It uses robots which are on the frontiers of explored space to guide the Swarm in unexplored space. An optimal path is created for the other robots to move optimally towards the frontier. For an exact description of the implementation of these algorithms please see \cite{mclurkin2007distributed}.
    \item \textbf{Random Walk}\cite{morlok2007dispersing}:\\
      This algorithm has 2 states: \emph{walking} and \emph{avoid obstacle}. In the \emph{walking} state each robot keeps going straight with an orientation which randomizes over a certain time interval until there is an object and switches to the \emph{avoid obstacle} state. In the case that it detects a possible collision, the robot changes orientation dramatically until the obstacle has been avoided and continues in the \emph{walking} state.
    \item \textbf{Follow Wall}\cite{morlok2007dispersing}:\\
      The follow wall algorithm has 4 states: \emph{find wall}, \emph{align to wall}, \emph{follow wall} and \emph{navigate corner}. The details of this algorithm will not be discussed, as there are quite a few major flaws in this algorithm. This algorithm does not take the existance of other robots into account and thus it is possible for robots to see each other as walls. The usage of this algorithm for dispersion is thus very limited and should be avoided.
    \item \textbf{Seek Open}\cite{morlok2007dispersing}:\\
      By calculating an average obstacle vector with support of distance censors, a vector in the exact opposite direction is calculated and the robot, will follow this vector. Depending on the magnitede of the vector, a new assesment will be done once the robot reaches the approximated location. This means that the robot does not need to take further care of collisions with walls, but it is possible for robots to run into each other, or other dynamically moving objects, unless collision avoindance is separetely implemented.
    \item \textbf{Fiducial} \cite{morlok2007dispersing}:\\
      By using a beacon system, every robot is able to get the relative location of other robots within a certain range. This information is the used to steer away from the other robots. If no robots are in range, the robot uses the \emph{Random Walk} algorithm.
    \item \textbf{Clique-Intensity}\cite{ludwig2006robotic}:\\
      This algorithm uses the connectivity in a cyclic graph for swarm robots to disperse the robots, by knowing their relative distance. By using multiple behaviours as many robots as possible try to stay connected to only one other robot, thus making a very spread out system, which still remains connected.
  \end{itemize}

  \begin{table}[H]
  \renewcommand{\arraystretch}{1.3}
  \caption{Overview of Common Dispersion Algorithms}
  \label{table_alg_dispersion}
  \centering
    \begin{tabular}{|l|l|l|l|}
    \hline
    \bfseries Algorithm & \bfseries Type & \bfseries Performance & \bfseries Scalability\\
    \hline
    \bfseries DFLF& Location-Free & Medium-High & High\\\hline
    \bfseries BFLF & Location-Free & High & High\\\hline
    \bfseries Directed Dispersion & Range-Based & Medium & High\\\hline
    \bfseries Random Walk& Range-Based & Low & Low\\\hline
    \bfseries Follow Wall& Range-Based & Low & Low\\\hline
    \bfseries Seek Open& Range-Based & Low & Medium\\\hline
    \bfseries Fiducial& Range-Based & Medium & High\\\hline
    \bfseries Clique-Intensity& Range-Based & High & High\\\hline
    \end{tabular}
  \end{table}

  \subsubsection{Problems}
  \textbf{Range-Based}

  Starting from the simpler Algorithms such as random walk and wall following, we show the problems that the dispersion technique has faced in the past and how the newer algorithms have solved these problems. The \emph{Random Walk} algorithm has to be one of the simpelest algorithms, the problem this algorithm faces though, is that dispersiveness of the robots does not guarantee uniform dispersion. The algorithm can be seen as a brute-force approach, it possibly achieves our end-goal, but it does not do this in an optimal and scalable way.

  \emph{Wall Following} is an algorithm which is used a lot in robotic swarms, however it is not very effective by itself and faces many problems with scalability.  One of the main problems with scalability is the fact that the robot does not distinguish other robots and actual walls. This causes an extereme amount of collisions and thus with large amounts of robots, this algorithm becomes near to useless. 

  The \emph{Seek Open} algorithm is a good example of the real usage of Range-Based navigation, due to the fact that it actually uses the magnitudes of the data provided and does calculations in order to find the best position to move towards. The problem that this algorithm faces however is that it should be implemented with a collision avoidance algorithm. The algorithm is not really adapted to work with other dynamic objects, such as other robots. It is therefore not a very scalable algorithm by itself.

  The \emph{Fiducial} algorithm uses the Random Walk algorithm. A problem with the Random Walk algorithm has been solved here: the beacon like system, prevents the robots from running into each other. The Fiducial algorithm does not have any specific problems, however it still is a brute-force approach and thus does not guarantee uniform dispersion.

  The main problem with the \emph{Clique-Intensity} algorithm is the fact that due to high amounts of noise in the wireless intensity signals there is a lot of uncertainty in some real world applications. In a perfect situation, the algorithm would also work near perfectly. The work in this area has mostly been theoretical, real-world application is very different compared to theoretical situations.\\

  \textbf{Location-Free}

  The \emph{BFLF} algorithm, requires the robots to travel less compared to the \emph{DFLF} algortihm. The DFLF algorithm is furthermore also more computationally expensive than the DFLF algorithm. There is no big difference further during the execution of both algorithms, and thus BFLF has the preference. The problems that these types of algorithms face are not theoretical, but are coming forth from the category that they're in: often it is impossible to know the exact location, even if it's relative. There are quite a few ways that people try to achieve to create a relative position grid, however it difficult if not impossible to achieve in many real world applications.\\

\subsubsection{Remaining-Problems}
  The remaining problems in dispersion algorithms can be generally categorized into range-based problems and location-free problems, since all of the algorithms that are in these categories, are facing similar problems.
  The focus needed for the range-based approach needs to be on the uniformity of the dispersion. So how can we guarantee uniformity when dispersing the robots.
  The focus needed for location-free approaches is: how do we actually implement this in real world applications. There are minor to no problems in theory, however to actually bring the relative positioning into a grid is quite difficult. Research in this area should be focussed on how to create these relative positioning grids into a dependable and accurate manner.