%!TEX root = ../../Bachelorseminar-RoboticSwarms.tex
We consider dispersion as one of the key sub-problems needed to be solved in Robotic Swarms.\cite{ugur2007dispersion,mclurkin2007distributed,ludwig2006robotic} Dispersion can be compared to the blanket coverage problem, which was defined by in one of the first papers written on the topic of swarm robots: \emph{Command Control for Many-Robot Systems}.\cite{gage1992command} The objective of blanket coverage is defined as ``achieving a static arrangement of elements that maximizes the detection rate of targets appearing within the coverage area.''\cite{gage1992command} The difference between dispersion and formation is often confused. While formation is trying to maintain explicitly specified spacing relationships, dispersion tries to find the best ``formation'' for the current environment. 

\subsection{Applications}
Due to the fact that dispersion is one of the fundamental sub-problems in robotic swarms, it has a great number of applications.
Some of these applications include, but are not limited to exploration, surveillance, military response, disaster response and planetary exploration.\cite{ludwig2006robotic,Penders2011,mclurkin2007distributed} 

\subsection{Algorithms}
In this section we will discuss a few categories of approaches which are used to solve the dispersion problem.
Starting from the simpler algorithms wich are based on the brute-force approach such as random-walk and wall-following, we show the problems that the dispersion technique has faced in the past and how the newer algorithms have solved these problems.
Further we will motivate as to why the algorithms were put into a specific category and we will discuss the scalability and performance for each type of algorithm.
For more information on the operation of the mentioned algorithms, please consult the references.

  \begin{table}[H]
  \renewcommand{\arraystretch}{1.3}
  \label{table_alg_dispersion}
  \caption{Overview of Common Dispersion Algorithms}
  \centering
    \begin{tabular}{|l|l|l|l|l|}
    \hline
    \bfseries Algorithm & \bfseries Approach & \bfseries Orientation & \bfseries Performance & \bfseries Scalability\\
    \hline
    \bfseries Random Walk & Brute-Force & Range-Based & Medium & High\\\hline
    \bfseries Follow Wall & Brute-Force & Range-Based & Low & Low\\\hline
    \bfseries Directed Dispersion & Inverse-Vector & Range-Based & Medium & High\\\hline
    \bfseries Seek Open & Inverse-Vector & Range-Based & Low & Medium\\\hline
    \bfseries Fiducial & Inverse-Vector & Range-Based & Medium & High\\\hline
    \bfseries DFLF & Graph-Theory & Location-Based & Medium-High & High\\\hline
    \bfseries BFLF & Graph-Theory & Location-Based & High & High\\\hline
    \bfseries Clique-Intensity & Graph-Theory & Range-Based & High & High\\\hline
    \end{tabular}
  \end{table}

\subsubsection{Brute-Force Approach}
Under the brute-force approach we categorize algorithms such as \emph{Random-Walk} and \emph{Follow-Wall} algorithms.\cite{morlok2007dispersing} 
The most notable being the \emph{Random-Walk} algorithm due to its simplicity and efficiency. 
The base of these algorithms are location-free, but often include collision detection which makes them range-based.
The scalibility of the algorithms in this category differs greatly.
Random-walk for instance is known as one of the most scalable algorithms, due to its independence of other robots in the swarm.
On the other hand the performance of the follow-wall algorithm deteriorates greatly when increasing the scale. 
This due to the fact that they can not easily distinguish between the wall and other objects and thus creates the possiblity for the swarm-bots to circle around each other.
This causes an extereme amount of collisions and thus with large amounts of robots, making the algorithm non-scalable and many times not fit for the situation.

The only drawback with the \emph{Random-Walk} algorithm is that it does not guarantee uniform dispersion.
Of course as with many other brute-force algorithms, the algorithm is also not optimal, but due to its simplicity it is often recommended to be used.
In situation in which low-energy consumption has a high priority, the usage of this algorithm is discouraged. Due to its non-optimalism it will use a lot of energy when trying to disperse in a uniform manner.

\subsubsection{Graph Theory Approach}
The algorithms which are inspired by graph theory can be divided into two sub-categories: algorithms inspired by tree search algorithms and algorithms inspired by connectivity in graphs.\\

In a paper on algorithms for rapidly dispersing robot swarms in unkown environments\cite{hsiang2004algorithms}, two tree search inspired algorithms were named.
The paper mentions two strategies: \emph{Depth-First Leader-Follower(DFLF)} Strategy and the \emph{Breadth-First Leader-Follower} Strategy.
The implementation of these algorithms are mainly considered as \emph{location-based} algorithms, due to the fact that they work with grids which require absolute or relative positioning.
The \emph{BFLF} algorithm, requires the robots to travel less compared to the \emph{DFLF} algortihm. The DFLF algorithm is furthermore also more computationally expensive than the DFLF algorithm.
Further, there are no big differences during the execution of both algorithms, and thus BFLF has the preference.
Due to the high amount of computation which needs to be done centrally, the scalibility of this type of algorithms are rather low.
Every robot has to be taken care of by one or multiple robots. By centralizing this computation, there is a greater risk of something going wrong compared to decentralized control.\\

The problems that these types of algorithms face are not theoretical, but are coming forth from the category that they are in. They are all location-based. Often it is impossible to know the exact location, even if it's relative, but this a part of the Localization problem, which has to be solved.

One of the algorithms regarding connectivity in graphs is the clique-intensity algorithm.\cite{ugur2007dispersion}
The Clique-Intensity algorithm is range-based since it measures and uses the distance between itself and other surrounding swarm-bots.
The performance and scalability of the algortihm is very high, due to decentralized control.

The Clique-Intensity algorithm faces problems due to the fact that there are high amounts of noise in the wireless intensity signals when used in real-world applications. 
This causes lots of uncertainty in some real world applications.
The work in this area has mostly been theoretical, real-world application is very different compared to theoretical situations.
The Focus should be put on the localization using wireless signals.\\

\subsubsection{Inverse-Vector Approach}
Some examples of the Inverse-Vector approach are the seek-open algorithm\cite{morlok2007dispersing}, the Fiducial algorithm\cite{morlok2007dispersing} and the Uniform Directed Dispersion(UDD) algorithm\cite{mclurkin2007distributed}.
Each of the algorithms basically sense where obstacles and other swarm-bots are relatively positioned and calculate a vector of that data. Afterwards they calculate the inverse of that vector and moves into that direction.
The \emph{Fiducial} algorithm has a advantage over the \emph{Seek-Open} algorithm: it uses a beacon like system, which prevents robots from running into each other and encourages uniformatiy of the distribution.
The seek-open and UDD algorithm use other distance measures mostly using ultrasonic sensors. In both cases the algorithm is a range-based algorithm, where every robot is able to get the relative location. 

It is important to stress that the algorithms should perform periodic checks to detect dynamic changes in the environment, such as other moving robots.


\subsubsection{Remaining-Problems}
The remaining problems in dispersion algorithms can be generally categorized into range-based problems and location-based problems.
All the algorithms that are in these categories are facing similar problems.
The focus needed for the range-based approach needs to be on the uniformity of the dispersion. So how can we guarantee uniformity when dispersing the robots.
In the location-based approach category focus needs to be put on the applicability for real-world applications.
There are minor to no problems in theory, however to actually bring the relative positioning into a grid is quite difficult.
Research in this area should be focussed on creating high-accuracy relative positioning grids.
