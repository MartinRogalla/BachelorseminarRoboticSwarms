%!TEX root = ../../Bachelorseminar-RoboticSwarms.tex
We consider dispersion as one of the key problems needed to be solved in Robotic Swarms.\cite{ugur2007dispersion,mclurkin2007distributed,ludwig2006robotic} Dispersion can be compared to the blanket coverage problem, which is defined by in one of the first papers written on the topic of swarm robots: \emph{Command Control for Many-Robot Systems}.\cite{gage1992command} The objective of blanket coverage is defined as ``achieving a static arrangement of elements that maximizes the detection rate of targets appearing within the coverage area.''\cite{gage1992command} The difference between dispersion and formation is often confused. While formation is trying to maintain explicitly specified spacing relationships, dispersion tries to find the best ``formation'' for the current environment. \\

Due to the fact that dispersion is one of the fundamental problems in robotic swarms, it has a great number of applications.
Some of these applications include, but are not limited to exploration, surveillance, military response, disaster response and planetary exploration.\cite{ludwig2006robotic,Penders2011,mclurkin2007distributed} 

\subsection{Algorithms}
In this section we discuss a few categories of approaches which are used to solve the dispersion problem.
Starting from the simpler algorithms which are based on the approach which includes randomness such as random-walk and wall-following, we show the problems that the dispersion problem has faced in the past and how the newer algorithms have solved these problems.
Further we motivate as to why the algorithms are put into a specific category and discuss the scalability and performance for each type of algorithm.
For more information on the operation of the mentioned algorithms, please consult the references.

  \begin{table}[H]
  \renewcommand{\arraystretch}{1.3}
  \label{table_alg_dispersion}
  \caption{Overview of Common Dispersion Algorithms}
  \centering
    \begin{tabular}{|l|l|l|l|l|l|}
    \hline
    \bfseries Algorithm & \bfseries Approach & \bfseries Range & \bfseries Location & \bfseries Performance & \bfseries Scalability\\
    \hline
    \bfseries Random Walk & Brute-Force & Range-Based & Location-Free & Medium & High\\\hline
    \bfseries Follow Wall & Brute-Force & Range-Based & Location-Free & Low & Low\\\hline
    \bfseries Directed Dispersion & Inverse-Vector & Range-Based & Location-Free & Medium & High\\\hline
    \bfseries Seek Open & Inverse-Vector & Range-Based & Location-Free & Low & Medium\\\hline
    \bfseries Fiducial & Inverse-Vector & Range-Based & Location-Free & Medium & High\\\hline
    \bfseries DFLF & Graph-Theory & Range-Based & Location-Based & Medium-High & Low\\\hline
    \bfseries BFLF & Graph-Theory & Range-Based & Location-Based & High & Low\\\hline
    \bfseries Clique-Intensity & Graph-Theory & Range-Based & Location-Free & Medium & High\\\hline
    \end{tabular}
  \end{table}

\subsubsection{Randomness Approach}
%TODO: Find a few more references talking about random walk and follow-wall concerning usage in Robotic Swarms
The algorithms in this category try to solve the dispersion problem with an approach which relies on randomness to perform the dispersion.
All of these algorithms are mainly location-free, but often include collision detection which makes them range-based.\\
The most pre-eminent algorithm in this category is the \emph{Random-Walk} algorithm due to its simplistic nature and efficiency. 
The \emph{Random-Walk} algorithm changes the robot's orientation randomly and moves it forward until an obstacle is detected. The algorithm then repeats the steps indefinetely.\cite{morlok2007dispersing}
The algorithm is very scalable, due to the simple set of steps that have to be performed and due to its non-interdependence of other robots in the swarm.
A minor drawback with the \emph{Random-Walk} algorithm is that it does not guarantee uniform dispersion.
Due to the reliance on randomness, the algorithm is not optimal. If low-energy consumption has a high priority in the system, we do not recommend the usage of this algorithm.
In all other situations which allow its usage the \emph{Random-Walk} algorithm is greatly recommended due to its satisfying performance and highly rated scalability.\\
Another algorithm which is also used for dispersion purposes is the \emph{Follow-Wall} algorithm.\cite{morlok2007dispersing}
Originally the algorithm is not random, but follows an exact set of rules depending on the environment. 
The algorithm is created for a non-dynamic environment and not for usage in robotic swarms.
The robot is unable to see the difference between robots and other obstacles. This creates the possibility for robots to constantly follow each other, while thinking that they're actually moving around a wall or other static obstacly.
The performance rating is low due to the meager amount of randomness and its unability to gurantee the uniformity of the dispersion. The scalability is also rated low, due to the performance deterioration when increasing the scale.\\


\subsubsection{Graph Theory Approach}
The algorithms which are inspired by graph theory can be divided into two sub-categories: algorithms inspired by tree search algorithms and algorithms inspired by connectivity in graphs.\\

In a paper on algorithms for rapidly dispersing robot swarms in unkown environments\cite{hsiang2004algorithms}, two tree search inspired algorithms were named.
The paper mentions two strategies: \emph{Depth-First Leader-Follower(DFLF)} Strategy and the \emph{Breadth-First Leader-Follower} Strategy.
The implementation of these algorithms are mainly considered as \emph{location-based} algorithms, due to the fact that they work with grids which require absolute or relative positioning.
The \emph{BFLF} algorithm, requires the robots to travel less compared to the \emph{DFLF} algortihm. The DFLF algorithm is furthermore also more computationally expensive than the DFLF algorithm.
Further, there are no big differences during the execution of both algorithms, and thus BFLF has the preference.
Due to the high amount of computation which needs to be done centrally, the scalibility of this type of algorithms are rather low.
Every robot has to be taken care of by one or multiple robots. By centralizing this computation, there is a greater risk of failure due to instability compared to decentralized control.\\

The problems that these types of algorithms face are not theoretical, but are coming forth from the category that they are in. They are all location-based. Often it is impossible to know the exact location, even if it's relative, but this a part of the Localization problem, which has to be solved.

One of the algorithms regarding connectivity in graphs is the clique-intensity algorithm.\cite{ugur2007dispersion}
The Clique-Intensity algorithm is range-based since it measures and uses the distance between itself and other surrounding swarm-bots.
The performance and scalability of the algortihm is very high, due to decentralized control.

The Clique-Intensity algorithm faces problems due to the fact that there are high amounts of noise in the wireless intensity signals when used in real-world applications. 
This causes lots of uncertainty in some real world applications.
The work in this area has mostly been theoretical, real-world application is very different compared to theoretical situations.
The Focus should be put on the localization using wireless signals.\\

\subsubsection{Inverse-Vector Approach}
Some examples of the Inverse-Vector approach are the seek-open algorithm\cite{morlok2007dispersing}, the Fiducial algorithm\cite{morlok2007dispersing} and the Uniform Directed Dispersion(UDD) algorithm\cite{mclurkin2007distributed}.
Each of the algorithms basically sense where obstacles and other swarm-bots are relatively positioned and calculate a vector of that data. Afterwards they calculate the inverse of that vector and moves into that direction.
The \emph{Fiducial} algorithm has a advantage over the \emph{Seek-Open} algorithm: it uses a beacon like system, which prevents robots from running into each other and encourages uniformatiy of the distribution.
The seek-open and UDD algorithm use other distance measures mostly using ultrasonic sensors. In both cases the algorithm is a range-based algorithm, where every robot is able to get the relative location. 

It is important to stress that the algorithms should perform periodic checks to detect dynamic changes in the environment, such as other moving robots.


\subsection{Discussion}
The remaining problems in dispersion algorithms can be generally categorized into range-based problems and location-based problems.
All the algorithms that are in these categories are facing similar problems.
The focus needed for the range-based approach needs to be on the uniformity of the dispersion. So how can we guarantee uniformity when dispersing the robots.
In the location-based approach category focus needs to be put on the applicability for real-world applications.
There are minor to no problems in theory, however to actually bring the relative positioning into a grid is quite difficult.
Research in this area should be focussed on creating high-accuracy relative positioning grids.
