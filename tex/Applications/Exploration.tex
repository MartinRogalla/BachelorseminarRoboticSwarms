%!TEX root = ../Bachelorseminar-RoboticSwarms.tex

Exploration is a technique in which a swarm of robots try to fully explore an environment, which is one of the fundamental problems faced in mobile robotics. 
The main goal is to minimize the overall exploration time while still exploring the whole environment. 
The main problem faced when trying to achieve this goal is finding appropriate target points for each individual robot so that they simultaneously explore different regions of the environment. \cite{burgard2005coordinated} \\
The exploration technique is found in many robotic swarms techniques, for example in \emph{path-finding}, \emph{collective transport} and \emph{surveillance}.
Practical applications that apply the exploration technique are for example rescue missions. \cite{Naghsh2008,Penders2011}
In this particular paper, a robotic swarm applying the exploration technique is used to assist navigation for firefighters, used in situations in which their vision is blocked by smoke and obstacles. 
A last example of an application is cleaning. \cite{wagner2008cooperative}
Here, the exploration technique is used to clean a surface with cleaning robots, that communicate in a swarm, as fast and as efficient as possible. 
Of course, exploration techniques are used in many more different robotic swarm applications, and is the building block for many different other techniques. 

%OLD
%Exploring an environment is one of the fundamental problems faced in mobile robotics. 
%The main goal is to minimize the overall exploration time and the main problem faced when trying to achieve this goal is finding appropriate target points for each individual robot so that they simultaneously explore different regions of the area \cite{burgard2005coordinated}. 
%Robotic swarm exploration can be used for real-world applications like rescue missions \cite{Naghsh2008,Penders2011}, surveillance \cite{Burkle2010} and cleaning \cite{wagner2008cooperative}.