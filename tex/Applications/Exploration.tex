%!TEX root = ../Bachelorseminar-RoboticSwarms.tex

\textbf{Description: }
\emph{Cleaning/exploring a component in an unkown area is a common problem which can be solved effectively by using a tobotic swarm. For example in \cite{wagner2008cooperative} a swarm of robots is placed in a certain component which has to be cleaned totally. The robots have limited visibility and can only see other robots within a certain distance. Main goal is to not disconnect the component that has to be cleaned to be sure that the component will be get fully cleaned in the end.} \\
  The main techniques that are described in \cite{wagner2008cooperative} are algorithms that preserve the connectivity of the spot to be cleaned and prevent the clustering of robots.
Exploring an environment is one of the fundamental problems faced in mobile robotics. The main goal is to minimize the overall exploration time and the main problem faced when trying to achieve this goal is finding appropriate target points for each individual robot so that they simultaneously explore different regions of the area \cite{burgard2005coordinated}. Robotic swarm exploration can be used for real-world applications like rescue missions \cite{Naghsh2008,Penders2011}, surveillance \cite{Burkle2010} and cleaning \cite{wagner2008cooperative}.