%!TEX root = ../../Bachelorseminar-RoboticSwarms.tex

Source localization is a technique in which a swarm of robots try to locate a specific point, often used to find sources of disturbance. 
The main goal of this technique is to find all sources of disturbance as fast as possible. 
Source localization is used in virtually any other robotic swarms technique, because in most techniques something has to be found, whether it is a source or an object. \\

Obviously, this means that this technique has many practical applications.
A few of these examples will be mentioned here. 
A first example is chemical plume tracing, in which localization is used to detect clouds of high density chemicals. \cite{zarzhitsky2005distributed}
Another example which looks a lot like chemical plume tracing, is radiation source search. \cite{bashyal2008human} The difference is that in this case the source of leaking radiation is searched for, and not only for high density clouds. 
A third example is searching for fire, used to assist fire-fighters in their every-day work. \cite{marjovi2009multi}
As can be seen, this technique can be used to look for all kinds of emission sources, if the robots have the right sensors installed. \cite{cui2004swarm}
Although many practical applications can be found, a large amount of the work done in this field is purely theoretical.
This is due to the fact that the price of these individual robots is still rather high and thus it is expensive to produce a swarm.


%Source localization is used in a large category of robotic-swarm applications. 
%Some of the topics include, but are not limited to: chemical plume tracing\cite{zarzhitsky2005distributed}, radiation source search\cite{bashyal2008human}, fire searching\cite{marjovi2009multi} and other types of emission source localization\cite{cui2004swarm}. %A large amount of the work done in this field is purely theoretical, this is due to the fact that the price of these individual robots is still rather high and thus it is expensive to produce a large quantity.
%The main techniques underlying these applications include: control, communication, path-planning and distribution.\cite{Li2012}
