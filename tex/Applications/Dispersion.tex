%!TEX root = ../../Bachelorseminar-RoboticSwarms.tex

Dispersion is a technique in which a swarm of robots try to cover an area as large as possible.  
Dispersive applications make interactive use of autonomous robots in exploratory settings, and is thus often used in combination with exploration techniques. 
The robotic swarms are capable of supporting and enhancing operations co-operatively and are optionally coordinated by a single human supervisor. 
There is one big difference between dispersion and exploration, although at first they seem very much alike. 
The difference is that in exploration (and mapping) the whole area has to be explored, but in dispersion an area as large as possible should be covered. 
This means that in dispersion, the ultimate goal is to cover the whole area as fast as possible, but the robotic swarm has a possibility to stop exploring when reaching its maximum coverage. \\

An example of an application is, among others, planetary exploration. \cite{ludwig2006robotic,Penders2011}
In this example, a swarm of robots tries to explore a part of the planet, but has no intent to explore the whole planet. 
So, a dedicated dispersion algorithm is used. 
Other examples of applications include military response and disaster response applications. 

%OLD
%Dispersive applications make interactive use of autonomous robots in exploratory settings. 
%The robotic swarms are capable of supporting and enhancing operations co-operatively and are optionally coordinated by a single human supervisor. 
%Possible real world applications include: military response, disaster response and planetary exploration\cite{ludwig2006robotic,Penders2011}
%The techniques required for these types of dispersion include, but are not limited to: \emph{foraging}, \emph{dispersion}, \emph{mapping}  and \emph{exploration}.\cite{morlok2007dispersing,Penders2011,mclurkin2007distributed} The foraging techniques are needed in order to give the swarm the ability to search and locate subjects. Dispersion is required in order for the swarm to navigate optimally and prevent conflicts in exploration. The mapping and exploration techniques are required to create a well constructed map of the explored areea, such that the human and other robots are aware of their surroundings, even if it is impossible to get a visual due to reduced visibility.\cite{morlok2007dispersing}
    