%!TEX root = ../../Bachelorseminar-RoboticSwarms.tex

  \emph{Dispersive applications make interactive use of autonomous robots in exploratory settings. The robotic swarms are capable of supporting and enhancing operations co-operatively and are optionally coordinated by a single human supervisor. Possible real world applications include: military response, disaster response and planetary exploration}\cite{ludwig2006robotic,Penders2011}
    The techniques required for these types of dispersion include, but are not limited to: \emph{foraging}, \emph{dispersion}, \emph{mapping}  and \emph{exploration}.\cite{morlok2007dispersing,Penders2011,mclurkin2007distributed} The foraging techniques are needed in order to give the swarm the ability to search and locate subjects. Dispersion is required in order for the swarm to navigate optimally and prevent conflicts in exploration. The mapping and exploration techniques are required to create a well constructed map of the explored areea, such that the human and other robots are aware of their surroundings, even if it is impossible to get a visual due to reduced visibility.\cite{morlok2007dispersing}
    