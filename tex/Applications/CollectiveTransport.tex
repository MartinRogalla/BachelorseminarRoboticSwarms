%!TEX root = ../Bachelorseminar-RoboticSwarms.tex

 \textbf{Description: } \emph{Transporting objects by robotic swarms has many potential applications in many settings, from agriculture to construction to disaster relief. Especially in dangerous settings like warzones or radio-active areas, robotic swarms can be a powerful tool to safely retrieve many objects. Especially because the robots used for this application are cheap to produce. }
For this application to work, each agent only has to know the target direction but does not have to know the object shape, weight, its own position or the position and number of other agents. This makes this application location-free and range-free. An extra invariant is that there must be enough robots to overcome the static friction of the object to be transported. Some of the techniques used in collective transport are: collective intelligence, decentralized strategy, path-finding. \cite{Rubenstein}
