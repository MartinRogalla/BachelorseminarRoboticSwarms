%!TEX root = ../../Bachelorseminar-RoboticSwarms.tex

Collective transport of objects is the technique of a swarm of robots locating an object and collectively moving the object to another place, like a homebase. 
This can be compared to a foraging technique; although this implies that a path is made to a certain place. 
This does not apply to all collective transport techniques. \cite{hoff2010two}  
It can be easily seen that localization is an important part of the collective transport technique, and thus has much overlap .\\
Transporting objects by robotic swarms has many potential applications in many settings, from agriculture to construction to disaster relief. 
Especially in dangerous settings like warzones or radio-active areas, robotic swarms can be a powerful tool to safely retrieve many objects. 
More so because the robots used for this application are cheap to produce. 


%For this application to work, each agent only has to know the target direction but does not have to know the object shape, weight, its own position or the position and number of other agents. This makes this application location-free and range-free. An extra invariant is that there must be enough robots to overcome the static friction of the object to be transported.  \cite{Rubenstein}
