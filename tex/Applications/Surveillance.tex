%!TEX root = ../Bachelorseminar-RoboticSwarms.tex

Surveillance is a technique in which a swarm of robots continuously patrol the same space, and simultaneously keep an eye out for abnormalities which are specified beforehand. Often this technique also includes formation, because patrolling robots should be in the same formation. Furthermore, in unknown locations mapping would be used, and in locating abnormalities a localization technique is often used. \\
Surveillance with a single robot is already a powerful tool, but with a swarm of robots it becomes even more effective. A larger area can be covered, local communication can be handled efficiently and depleted robots can interchanged with other robots. The applications of swarm surveillance are diverse, and are already used in a vast range of applications like agricultural practices, police surveillance, inspecting unreachable locations, patrol missions and reconnaisance tasks. \cite{Burkle2010} \\
The application can be implemented in many different ways, but is able to work location-free and range-free, but in some applications the location is already known and in others it can be helpful for a robot to know the exact location of the other robots of the swarm. In some applications the process isn't even completely autonomous; a human user could be in a ground control station. Some of the techniques used in swarm surveillance are: dispersion, localization, distributed communication and exploration. 