%!TEX root = ../../Bachelorseminar-RoboticSwarms.tex
Collaborative Mapping in Robotic Swarms is a problem in which a robotic swarm collaboratively tries to map the surrounding environment. 
The Collaborative Mapping problem is interdependent with the Localization problem. Due to its dependency on \emph{relative} or \emph{absolute} location, the Collaborative Mapping problem requires a techniques to solve the Localization problem.\\

Collaborative Mapping of an unknown environment is very useful in hazardous and unaccessable environments.\cite{hardin2004modified}
Due to the ability of delegation in Robotic Swarm, the interdependence between robots within the robotic swarm is greatly reduced.
This greatly improves the stability of the mapping problem in comparison with using single robots. This greatly increases the reliability in hostile situations.\\

In \emph{Distributed multi-robot coordination in area exploration} \cite{sheng2006distributed} a model is described in which the robots have a limited communication range and have full knowledge of their location.
By continuously looking for the closest undiscovered area in combination with a nearest measure it tends to explore the complete area by staying together and continiously updating each others maps.

\subsubsection{Algorithms}
In this section we will focus on collaborative mapping algorithms for usage in Robotic Swarms.

\begin{itemize}
  \item \url{http://robots.stanford.edu/papers/thrun.map3d.pdf}, \url{http://robots.stanford.edu/papers/thrun.maps-multi.pdf}
  \item \url{http://link.springer.com/chapter/10.1007/978-3-540-30552-1_6}
  \item \url{http://ieeexplore.ieee.org/stamp/stamp.jsp?tp=&arnumber=1014804}
  \item \url{http://books.google.nl/books?hl=en&lr=&id=3c9w6XEUxIMC&oi=fnd&pg=PA1&dq=collaborative+mapping+robots&ots=ZB-hzyo0zK&sig=8oT_KIZnGViutyzkds4PvC-YwEE#v=onepage&q=collaborative%20mapping%20robots&f=false}
\end{itemize}

  \begin{table}[H]
  \renewcommand{\arraystretch}{1.3}
  \label{table_alg_mapping}
  \caption{Overview of Common Mapping Algorithms}
  \centering
    \begin{tabular}{|l|l|l|l|}
    \hline
    \bfseries Algorithm & \bfseries Type & \bfseries Performance & \bfseries Scalability\\
    \hline
    \bfseries DFLF& Location-Free & Medium-High & High\\\hline
    \end{tabular}
  \end{table}

