%!TEX root = ../../Bachelorseminar-RoboticSwarms.tex
Collaborative Mapping in robotic swarms is a problem in which a robotic swarm collaboratively tries to map the surrounding environment. 
The Collaborative Mapping problem is interdependent with the Localization problem.\cite{thrun2000real}
Due to its dependency on \emph{relative} or \emph{absolute} location, the Collaborative Mapping problem requires a solution for the Localization problem.
Furthermore algorithms for solving the Localization problem, given a solution to the Mapping problem, have also been created.\cite{borenstein1996navigating} 
This show exactly the interdependency of the two problems.
The complexity of the problem greatly increases once there is no solution for either problem.
So either the map is not given, or the exact location of the bot is not given.\\

Collaborative Mapping has many applications.
For example, its usage is very convenient when trying to map an unknown environment in hazardous and unaccessable environments.\cite{hardin2004modified}
Due to the ability of delegation in robotic swarm, the interdependence between robots within the robotic swarm is greatly reduced.
This greatly improves the stability of the mapping problem in comparison with using single robots and increases the reliability in hostile situations.\\

\subsection{Algorithms}
In this section we will focus on collaborative mapping algorithms for usage in Robotic Swarms.

%Problems in cyclic environments see paper of thrun\ldots solved by EM approach. Can't be run in real-time. Only 2d maps, not yet 3d maps.
\subsubsection{A Probablilistic Approach}
The EM
\subsubsection{Likelihood Maximization Approach!!Inapplicable for cyclic environments}
\subsubsection{Posterior Approach}
\subsubsection{Monte Carlo Localization}
\subsubsection{Another Approach}
\url{http://www.aaai.org/Papers/AAAI/2000/AAAI00-131.pdf}

The Collaborative Localization Algorithm

Dynamic Task Allocation Algorithm

\url{http://ieeexplore.ieee.org/stamp/stamp.jsp?tp=&arnumber=1067998}


\begin{itemize}
  \item \url{http://robots.stanford.edu/papers/thrun.map3d.pdf}, \url{http://robots.stanford.edu/papers/thrun.maps-multi.pdf}
  \item \url{http://link.springer.com/chapter/10.1007/978-3-540-30552-1_6}
  \item \url{http://ieeexplore.ieee.org/stamp/stamp.jsp?tp=&arnumber=1014804}
  \item \url{http://books.google.nl/books?hl=en&lr=&id=3c9w6XEUxIMC&oi=fnd&pg=PA1&dq=collaborative+mapping+robots&ots=ZB-hzyo0zK&sig=8oT_KIZnGViutyzkds4PvC-YwEE#v=onepage&q=collaborative%20mapping%20robots&f=false}
\end{itemize}

  \begin{table}[H]
  \renewcommand{\arraystretch}{1.3}
  \label{table_alg_mapping}
  \caption{Overview of Common Mapping Algorithms}
  \centering
    \begin{tabular}{|l|l|l|l|}
    \hline
    \bfseries Algorithm & \bfseries Type & \bfseries Performance & \bfseries Scalability\\
    \hline
    \bfseries DFLF& Location-Free & Medium-High & High\\\hline
    \end{tabular}
  \end{table}

