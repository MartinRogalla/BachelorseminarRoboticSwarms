%!TEX root = ../Bachelorseminar-RoboticSwarms.tex

Swarm robotics has become a prominent and promising research area in the recent years. 
It has great potential use for a large variety of applications, some of which have already been successfully implemented. 
We want to provide a global overview of the main problems found in the research area of robotic swarms. 

Many articles already exist which give an overview of applications and used practices in this area, but often do not explain the problems underlying these practices. 
Therefore, we focus on providing a problem-oriented overview of the robotic swarms, while also providing a general overview of best practices and solutions of these problems.\\

We start by defining some terminology, since some of the terms used in robotic swarms are ambiguous and can be interpreted in different ways.
We define a swarm as a scalable network of robots which consists out of more than two robots.
Furthermore, we only consider robotic swarms in which every robot has some form of distributed intelligence.
An exception is of course when a swarm of multiple robots is controlled by one control station.
Because the swarm robots have to communicate either directly or indirectly with each other, the swarm will still have some form of distributed intelligence to function.
% Thus, according to our original definition, we still regard it as a swarm. \\
\\
Robotic swarm algorithms can roughly be characterized by their location type and their information type. Algorithms can then respectively be either \emph{location-based} or \emph{location-free} and either \emph{range-based} or \emph{range-free}:
\begin{description}
	\item[Location-free] Robots have no knowledge of their absolute location but may keep track of their relative location.
	%A robotic swarm is \emph{location-free} if the swarm has no knowledge of the boundaries of the location it is in, whether it is provided at the beginning or is actively searched for during the execution of the algorithm. 
	\item[Location-based] Robots have perfect knowledge of their absolute location.
	%A robotic swarm is \emph{location-based} if each individual robot in the swarm has the knowledge of its absolute or relative location.
	\item[Range-free] Robots do not communicate or communicate via some kind of central base.
	%A robotic swarm is \emph{range-free} if each robot can detect the presence of other nearby robots or obstacles, but does not store or measure the distance towards the other object.
	\item[Range-based] Robots communicate within predetermined range.
	%A robotic swarm is \emph{range-based} if each robot in the swarm keeps track of the exact distance between itself and the other robots in the swarm or obstacles. 
\end{description}

In the following sections, we compare the algorithms by two characteristics. 
The first characteristic is \emph{scalability}, by which we mean the ability of maintaining performance when the population in the robot swarm is increased. 
The second is \emph{performance}, by which we mean the general efficiency. 
We define efficiency separately for each problem before comparing the algorithms of that problem. 
We do this because the solutions have different ways of expressing efficiency for each problem.\\

A composite problem is a problem composed of multiple main problems in such a way that these main problems influence the working of the solutions. 
Although such a composite problem can be singled out, a lot of the main problems have some form of overlap too, although not with significant impact. 
The main problems Dispersion and Source Localization both include the Formation problem, and the Collective Transport is composed of the Formation problem among others.
The Exploration problem is more of an extension to the Dispersion problem. 
This relationship is shown in Figure~\ref{fig:ProblemsOverview}. \\
\begin{figure*}
  \centering
  \begin{tikzpicture}[->,>=stealth,shorten >=2pt,auto,node distance=3.5cm,
    thick,main node/.style={fill=white,draw,font=\sffamily}]
    \node at (7.7,0.7) {main problems};
    \node at (7,-4.3) {composite problems};
    \draw[fill=white,dashed] (-1.2,-2) rectangle (8.8,0.5);
    \draw[fill=white,dashed] (8.8,-4) rectangle (5.2,-3);
    \node[main node] (1) {Exploration};
    \node[main node] (2) [below= 0.9cm of 1] {Dispersion};
    \node[main node] (3) [right of=1] {Formation};
    \node[main node] (4) [right of=3] {Source Localization};
    \node[main node] (5) [below of=4] {Collective Transport};

    \path[every node/.style={font=\sffamily\small}]
      (1) edge [right] node[left] {} (2)
      (4) edge [right] node[left] {} (3)
      (5) edge [right] node[left] {} (4)
      (5) edge [right] node[left] {} (3)
      (1) edge [right] node[left] {} (3);
  \end{tikzpicture}
  \caption{Problem Composition Overview} \label{fig:ProblemsOverview}
\end{figure*}



The remainder of this paper is then structured as follows. 
In Section~\ref{sec:Formation} until Section~\ref{sec:Localization} we define the main problems in robotic swarms. 
Then, in Section~\ref{sec:CollectiveTransport}, we discuss the composite problem Collective Transport. 
For each problem we mention the possible real-life applications, their subproblems and the underlying algorithms of the solutions.
After that we discuss the characteristics of each algorithm, the corresponding (dis)advantages and where possible some remaining problems.
For more information on the operation of the mentioned algorithms, you can check the references provided. 
Finally we briefly discuss our observations in Section~\ref{sec:Discussion}.



