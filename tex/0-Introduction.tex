%!TEX root = ../Bachelorseminar-RoboticSwarms.tex
As robots become smaller and easier to produce, interest for robotic swarms is generated. Many possible applications for robotic swarms exist and it is certain many more will follow, indicated by a growing amount of paper written about robotics at for instance the AAMAS (International Conference on Autonomous Agents and Multiagent Systems). \cite{Amigoni2014} Many different applications and techniques exist in the field of robotic swarms. This paper aims to deliver a concise review of these applications and techniques. \\

To avoid confusion, some terminology will be defined. Robotic swarm applications can roughly be characterised by two attributes; they are either \emph{location-based} or \emph{location-free}, or they are either \emph{range-based} or \emph{range-free}.\todo[inline]{Does one exclude the other\ldots location-free$\to \neg$range-free?} The definitions of these attributes may be interpreted ambiguously, which is why we will define it here. The definitions are:

  \begin{itemize}
    \item A robotic swarm is \emph{location-free} if the swarm has no knowledge of the boundaries of te location it is in, whether it is provided at the beginning or is actively searched for during the execution of the algorithm. 
    \item A robotic swarm is \emph{location-based} if the swarm has the knowledge of predefined boundaries of the location it operates in, whether provided at the beginning of the execution of the algorithm or if it is actively searched for. 
    \item A robotic swarm is \emph{range-free} if each robot can detect, but does not store or measure the distance, other nearby robots in the swarm or nearby obstacles
    \item  A robotic swarm is \emph{range-free} if each robot in the swarm keeps track of the exact distance between itself and the other robots in the swarm or obstacles. 
  \end{itemize}

  \missingfigure[figwidth=\textwidth,caption={Venn Diagram of types of navigation.}]{Venn Diagram of types of navigation.}

In order to emphasize the importance of the connection between the technology and its applications, a top-down approach is used for this survey. Thus,  in this paper we will first review a few different application fields, after which a few examples of applications in these fields will be given. In the second part of the paper we will discuss the most used techniques for these applications and the algorithms behind these techniques. \\
  \subsection{Cleaning}
    Cite: \cite{Wagner2008} \cite{Altshuler2005}
  
  \subsection{Swarm-Assisted Fire Fighting}
  \textbf{Description: }\emph{Swarm-Assisted Fire Fighting makes interactive use of autonomous robots in fire emergency settings. These swarms of robots are capable of supporting and enhancing fire fighting operations co-operatively with each other and are coordinated by a single human supervisor.}\cite{Naghsh2008,Penders2011}\\

  The techniques required for Swarm-Assisted Fire Fighting include, but are not limited to: \emph{foraging}, \emph{formation}, \emph{mapping} and \emph{exploration}.\cite{Naghsh2008,Penders2011}\todo[inline]{Check more sources and verify techniques} The foraging techniques are needed in order to give the swarm the ability to search and locate victims. Formation is required in order for the swarm to navigate optimally and prevent conflicts in exploration. The mapping and exploration service are required to create a well constructed map of the explored area, such that the human and other robots are aware of their surroundings, even if it is impossible to get a visual due to reduced visibility caused by smoke.

  \subsection{Toxic Emition Source Discovery by Robotic-Swarms}
  \textbf{Description: }\emph{Due to the ambiguity of this name, there have been many applications which have aimed to achieve this, from nuclear spills and oil spills to fire-origins. Much of this however is theoretical work, due to the fact that the price of these individual robots is still rather high.}

  Some of the techniques used in toxic source discovery are: control, communication and distribution.\cite{Li2012}
  
  \subsection{Draft}
  Give an overview of real-world applications possible with Robotic Swarms. A list of possible applications:
    \begin{enumerate}
      \item Cleaning
      \item Space Exploration (swarm of Mars rovers)
      \item Rescue Missions
      \item Treacherous Radioactive Survey
      \item Survey and cleanup of Toxic Spills
      \item Surveillance
    \end{enumerate}
  \textbf{Categories}
    \begin{itemize}
      \item Region Covering
      \item Dangers
      \item Scaling in time
      \item Redundancy
    \end{itemize}
  