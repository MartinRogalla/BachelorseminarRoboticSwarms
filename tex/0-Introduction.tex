%!TEX root = ../Bachelorseminar-RoboticSwarms.tex
As robots become smaller and easier to produce, interest for robotic swarms is generated. Many possible applications for robotic swarms exist and it is certain many more will follow, indicated by a growing amount of paper written about robotics at for instance the AAMAS (International Conference on Autonomous Agents and Multiagent Systems). \cite{Amigoni2014} Many different applications and techniques exist in the field of robotic swarms. This paper aims to deliver a concise review of these applications and techniques. \\

To avoid confusion, some terminology will be defined. Afterwards, in order to emphasize the importance of the connection between the technology and its applications, a top-down approach is used for this survey. Thus, in this paper there will first be decided on the terminology of the research area, after which some applications will be given.  In the second part of the paper we will discuss the most used techniques for these applications and the algorithms behind these techniques. At the end a final overview and a discussion will be given.\\