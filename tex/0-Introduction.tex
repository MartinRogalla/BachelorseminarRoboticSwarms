%!TEX root = ../Bachelorseminar-RoboticSwarms.tex
The robotic swarm field is growing as robots become smaller and easier to produce.
The usage of robotic swarms compared to single robots has numerous advantages. 
These advantages include a higher efficiency, shared knowledge which decreases mutual dependencies and thus improves realiability.
Many new applications for robotic swarms exist creating a strong growth in the field.
This is indicated by a growing amount of papers written about robotics at for instance the AAMAS \footnote{International Conference on Autonomous Agents and Multi-Agent Systems}.\cite{Amigoni2014}\\

To avoid confusion we will start off by defining some terminology, because the terms are quite ambiguous and can be interpreted in many different ways.
In Section \ref{sec:Formation} until Section \ref{sec:Path-planning} we will define the problems in robotic swarms, where we mention the applications and their underlying solutions.
Finally we in will give a brief discussion on our observations in Section \ref{sec:Discussion}.\\

In this survey, we will only consider something a swarm when the amount of robots is greater than two and the amount of robots is scalable and uses some form of distributed communication.
Furthermore, we will only consider robotic swarms in which each robot is not controlled individually.
They should have some form of distributed intelligence.
An exception is of course when a swarm of multiple robots is controlled by one control station.
This swarm will still have some form of distributed intelligence to function and thus is considered a swarm.  \\

Robotic swarm applications can roughly be characterised by two attributes. They are either \emph{location-based} or \emph{location-free}, or they are either \emph{range-based} and  \emph{range-free}.
A location-free approach does not exclude a range-free approach and vice-versa. 
The definitions are:

  \begin{itemize}
    \item A robotic swarm is \emph{location-free} if the swarm has no knowledge of the boundaries of the location it is in, whether it is provided at the beginning or is actively searched for during the execution of the algorithm. 
    \item A robotic swarm is \emph{location-based} if each indivdual robot in the swarm has the knowledge of its absolute or relative location.
    \item A robotic swarm is \emph{range-free} if each robot can detect the presence of other nearby robots or obstactles, but does not store or measure the distance towards the other object.
    \item A robotic swarm is \emph{range-based} if each robot in the swarm keeps track of the exact distance between itself and the other robots in the swarm or obstacles. 
  \end{itemize}

