%!TEX root = ../Bachelorseminar-RoboticSwarms.tex

Collective transport of objects is the problem in which a swarm of robots locates a payload and collectively moves the payload to another place, like a home base. 
This problem is a composite problem, consisting of the already mentioned problems formation and source localization. 
Still, we want to discuss this problem because it is an important problem in the research field of robotic swarms.
%This can be compared to the path-planning problem; although this implies that a path is made to a certain place, which is not the case in the collective transport problem. \cite{hoff2010two}  
%The localization problem is an important part of the collective transport problem and has much overlap.\\
Transporting objects by robotic swarms has many potential applications in many settings, from agriculture to construction to disaster relief. 
Especially in dangerous settings like war zones or radio-active areas, robotic swarms can be a powerful tool to safely retrieve many objects. 
For example, recently Amazon, a large online retailer, announced it would make use of unmanned flying robots to deliver parcels at everyone's door. \footnote{Amazon's PrimeAir service, to be released sometime in 2015. www.amazon.com/b?node=8037720011}

\subsection{Algorithms}

Collective transport can be separated into two subproblems: providing safe and reliable transport and moving a payload to its destination.
These respectively correspond to the formation problem and the source localization problem.

%If a homogeneous swarm is controlled by a single ground station, the algorithm is not considered with finding a payload.
%In algorithms in which swarms are completely autonomous, the focus is on finding the payload and transporting it to its destination.
%Instead it is focused on safe, accurate and scalable transport. 
%So, we can categorize these subproblems as user-controlled transportation problems and autonomous transportation problems.
%In user-controlled transportation, the location the swarm operates in is known, so the algorithms used are location-based.
%In autonomous transportation, the location is not known and the algorithms are location-free.

When considering the formation problem, the algorithm is not considered with finding a payload. 
Then, we can conclude that the swarm in this algorithm is not completely autonomous and that the whole swarm is controlled by a single controller.
This implies that algorithms used for the formation problem are location-based, because the location is known by the controller. \cite{Michael2011cooperative,mas2012object}
When considering the source localization problem, the algorithm is not concerned with transporting the payload. 
Then, we can conclude the swarm in this algorithm works autonomously. 
This implies that the algorithms used for the source localization problem are location-free, because the location is now known by the controller. \cite{Rubenstein2013collecive, sugawara2012object}

\subsubsection{Formation transport}
%A robot, especially a flying robot, can only carry a certain weight, which poses a problem. 
%But with a robotic swarm of robots it is possible to transport larger weights in a scalable system, adding swarm robots until you can carry the desired weight. 
%So, a swarm of robots is useful for transporting large and/or heavy objects. 
There are two different properties important for formation transport: performance and scalability. 
We define \emph{performance} as in how well the robots can stay in formation in dynamic environments. \\
%To ensure safe and stable transport, swarm robots have to be coordinated in such a w ay that the payload can be manipulated to a desired position and orientation.

The first algorithm we discuss is the \emph{Aerial Equilibrium} algorithm. \cite{Michael2011cooperative}
This algorithm for aerial transport is implemented by attaching cables from quadrocopters to the payload.
Then, each robot calculates its own movement in the swarm by a mathematical model. 
Because the swarm robots do not use local communication but get instructions from a user, this algorithm can be defined as a range-free algorithm. 
A change in the position of a robot is quickly communicated through the mathematical model, resulting in a good performance.
Nevertheless, the scalability is not that good, because adding multiple robots will quickly result in robots getting in each others way.
This is because the chance of touching each others rope increases with every robot added. \\

%An algorithm for such transport is given in the paper Cooperative manipulation and transportation with aerial robots.
%This paper formulates general conditions for the payload equilibrium but can be scaled to many swarm robots. 
%The mathematical model that is defined in the paper is tested with three quadrocopters and offers a stable transportation for objects, with a degree of freedom related to the amount of robots used and can thus be implemented in the field.

The second algorithm we discuss is called \emph{Cluster Space Control} algorithm. \cite{mas2012object}
It uses a multi-robot formation control framework called cluster space control.
It is utilized to control a swarm of four four-wheeled robots.
A user using this type of transportation uses a joystick to input user controls to the whole swarm, after which the centralized control framework specifies the formation and the position and shape of the swarm.
Because of the centralized communication there is no local ranged communication, so this algorithm is range-free. 
This control framework allows the user to effectively transport large objects with a scalable set of robots. 
The performance for this algorithm is very high, because the framework responds well to changes in the environment. 
The scalability is low, because the framework has to hold connections to every robot. \\

%The algorithm can be used for transporting large objects on the ground, in the water and in the air.
%necessary to coordinate the motions of the robots in the group. 
%This is because control approach used is called cluster space control, and is utilized to control swarm of four four-wheeled robots. 

These two algorithms solve the same problem, namely the problem of transporting large and/or heavy objects, in two entirely different ways. 
They are both range-free and location-based, but differ mostly in scalability. 
As both algorithms have centralized communication, scalability is low for both.
We summarize the properties in the Table~\ref{table_alg_coltrans1}.

  \begin{table}[H]
  \renewcommand{\arraystretch}{1.3}
  \centering
\scalebox{0.85}{
  \begin{tabular}{|l|l|l|l|l|}
    \hline
    \bfseries Algorithm & \bfseries Range &\bfseries Location &  \bfseries Performance & \bfseries Scalability\\
    \hline
     Aerial Equilibrium & Range-free & Location-based & Medium & Low\\\hline
     Cluster Space Control & Range-free & Location-based & High & Low\\\hline
    \end{tabular}
    }
  \caption{User-controlled swarm transportation algorithms}
  \label{table_alg_coltrans1}
  \end{table}

In the first algorithm, each swarm robot individually calculates its position to the other robots and the payload and the resulting tension on the payload. 
This way, the rotation and position can be controlled by the position and tension of each robot. 
With the second algorithm, that uses the cluster \emph{cluster space control}, the calculations are not done individually. 
Instead, the framework calculates the position of each robot and the orientation of the payload. 
Therefore these methods both provide stability to the transported payload but in two completely different ways. \\

The problems these algorithms have are that the robots should be coordinated to rely on the location of other robots.
With the mathematical model these are calculated through the tension of the cable attached to each robot, but in the framework it is calculated centrally. 
Problems that are not treated in these articles is how these swarms can locate a payload and autonomously can deliver it to a location. 
Another problem for which no solution is provided is that when these swarms grow to account for heavier objects, the robots will get in each other's way. 
Especially considering transportation, a technique in which every robot must latch on to a payload, this can be hard to practically scale. 

\subsubsection{Source localization transport}
The source localization problem consists of finding the payload and transporting it to its destination.
This subproblem can again be divided in four smaller problems: finding the payload, grabbing the payload, finding target location, moving the payload to the location.
By defining these subproblems, the problem is easier to solve and easier to explain each algorithm. 
The \emph{performance} is defined in how fast the payload is moved to the destination. \\

The first algorithm we discuss is the flocking algorithm. \cite{Rubenstein2013collecive}
Simply put, it locates the payload through a flocking algorithm and let the swarm robots push the payload to the goal location. 
First every robot looks if it can see the payload location. If so, move towards the location. 
If not, the robot compares its own heading with the other robots nearby and corrects it until the robot finds the payload location.
In this paper, the payload location and the goal location are found with light sensors, communicates headings with infrared sensors and uses bump sensors to avoid obstacles. 
The robot latches on to the payload with a velcro strip.
After finding the payload, the robot pushes from an arbitrary side, calculating the amount of force needed in relation to the middle point of mass with its bump sensors and transporting it to the goal location with its light sensor.  
And because the robots latch onto an arbitrary side, scaling of the swarm is extremely effective. \\ 

%Each robot uses one of two behaviors.
% A robot that can see the goals aligns itself towards the goal while moving forward.
% A robot that cannot see the goal uses its IR sensors to determine the heading difference between itself and its neighbors.

%Another widely used algorithm for transporting objects is the \emph{pheromone} algorithm. 
%This algorithms relies on leaving a trail of marks that help guide other swarm robots. 
%Although this is algorithm can be used for foraging, it can also be used for retrieving a single payload leading many robots to transport a single payload.
%This algorithm is used in the paper Cooperative Transportation by Swarm Robots Using Pheromone Communication. \cite{fujisawa2013cooperative}
%In this paper, this algorithm is tested with swarm-bots which can release and detect spots of ethanol left on the ground, indicating a spot they have to follow. Although the pheromone algorithm implies that pheromones should be used, different varieties of markings can be used, including wireless sensor beacons, drops of paint and even objects. This algorithm is particularly useful for guiding many robots to a payload, if for example the payload is very heavy or if many robots are needed for accurate and/or stable transportation. This algorithm is range-based; it calculates the exact distance to a trail. This algorithm is difficult to implement in dynamic environments, as made marks can dissapear. \\

The following algorithm uses very simple robots, which is desirable. \cite{sugawara2012object}
It is based on granular convection, also known as the Brazil Nut Effect. 
The algorithm can be implemented in three different ways: (1) A swarm with homogeneous robots with no explicit communication, 
(2) a swarm of heterogeneous robots composed of robots that change their direction with two different probabilities, 
and (3) a swarm of heterogeneous robots that uses local communication to adjust the fraction of robots that change their direction with higher probability. 

Each implementation has a higher performance than the last one, but also increase in communication. \\ 
The way this algorithm works is as follows. 
The goal of the swarm robots is to transport the object to the goal location. 
Each swarm robot vibrates with a random force. The goal location outputs a repulsive force. 
As the robots are randomly placed around the payload and the goal location, the payload gets kicked around by the vibrating swarm robots.
This continues until the payload is kicked to the target location.
When the swarm increases, the amount of time it costs to transport the payload significantly decreases. 
With heterogeneous robots, there is more communication between robots to organize the vibrations of the robots. \\
%With heterogeneous robots, there is more communication between robots to lower the density of the robots around the payload, because slows the process of moving the payload.\todo{SB: I don't get this.}

The algorithms we describe are all location-free, as mentioned earlier. 
They are also range-free, working only with local communication.
They also have a high scalability, because robots can easily be added to such an extent that negative side-effects are not reached quickly.
With the granular convection method, adding local communication and using heterogeneous robots increases the performance a lot. 

  \begin{table}[H]
  \renewcommand{\arraystretch}{1.3}
  \centering
\scalebox{0.85}{
  \begin{tabular}{|p{7cm}|l|l|l|l|}
    \hline
    \bfseries Algorithm & \bfseries Range & \bfseries Location & \bfseries Performance & \bfseries Scalability\\
    \hline
    Flocking & Range-free & Location-Free & High & High\\\hline
    %Pheromone & Range-based & Location-Free & Medium & Medium\\\hline
    Homogeneous granular convection & Range-free& Location-Free & Low  & High\\\hline
    Heterogeneous granular convection & Range-free& Location-Free & Medium & High\\\hline
    Heterogeneous granular convection with \newline local communication & Range-free& Location-Free & High & High\\\hline

    \end{tabular}
    }
    \caption{Overview of Collective Transport Algorithms}
  \label{table_alg_coltrans2}
  \end{table}


%A problem these algorithms ran into was how to locate the payload, how to move the payload and how to find and reach the goal location. 
%The main problem is locating the payload, something which the granular convection algorithm actually never does.  
%What these three algorithms have in common is that the transportation process speeds up significantly when adding more swarm robots, but reaches a cap when it becomes ineffective due to maximal density. 
%For this reason, small, easy and cheap to produce swarm robots are required. 
%The granular convection algorithm has the cheapest robots, but is not necessarily the best.
%This is because when the robots become slightly larger and more complicated, more sophisticated and faster algorithms could work.\todo{SB: Whole alinea is formulated a bit vague and is repetitive compared to previous section, take a look at it.} \\
%These three algorithms are very different from each other, but they all serve a common goal. 

In the mentioned algorithms, moving the payload is mostly done by pushing. 
In real-life applications this would not always be a good option and a more sophisticated method of attaching should be used. 
When considering search-and-rescue operations for example, a solution to effectively grab the person to be rescued should be thought of. 
This is a problem that is yet to be identified in these algorithms and poses a good question for the future.
