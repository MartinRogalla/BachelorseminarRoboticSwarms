%!TEX root = ../../Bachelorseminar-RoboticSwarms.tex

Collective transport of objects is the problem in which a swarm of robots locates a payload and collectively moves the payload to another place, like a homebase. 
This can be compared to the foraging problem; although this implies that a path is made to a certain place, which is not the case in the collective transport problem. \cite{hoff2010two}  
The localization problem is an important part of the collective transport problem and has much overlap .\\
Transporting objects by robotic swarms has many potential applications in many settings, from agriculture to construction to disaster relief. 
Especially in dangerous settings like warzones or radio-active areas, robotic swarms can be a powerful tool to safely retrieve many objects. 
For example, recently Amazon, a large online retailer, announced it would make use of unmanned flying robots to deliver parcels at everyone's door. (www.amazon.com/b?node=8037720011) \\

Collective transport can be seperated into two sub-problems:
\begin{itemize}
	\item  providing safe and reliable transport
	\item moving a payload to its destination.
\end{itemize}
These two different problems will be handled separately because these are two entirely different problems that together compose the whole collective transport problem. 

In algorithms in which swarms are completely autonomous, the focus is on finding the payload and transporting it to its destination.
If a homogeneous swarm is controlled by a single ground station, the algorithm is not considered with finding a payload, but is instead focused on safe, accurate and scalable transport. 
So, we can categorize these problems as user-controlled transportation and autonomous transportation.
In user-controlled transportation, the location the swarm operates in is known, so the algorithms used are location-based.
In autonomous transportation, the location is not know and the algorithms are location-free. \\

\subsubsection{User-controlled swarm transportation algorithms}
A robot, especially a flying robot, can only carry a certain weight, which poses a problem. 
But with a robotic swarm of robots it is possible to transport larger weights in a scalable system, adding swarm robots until you can carry the desired weight. 
So, a swarm of robots is useful for transporting large and/or heavy objects. 
There are two different properties important for user-controlled swarm transportations: safety and scalability. 
To ensure safe and stable transport, swarm robots have to be coordinated in such a way that the payload can be manipulated to a desired position and orientation.
An algorithm for such transport is given in the paper Cooperative manipulation and transportation with aerial robots. \cite{Michael2011cooperative}, which we will call the \emph{Aerial Equilibrium} algorithm. 
In this paper, a scalable solution for aerial transport is presented by attaching cables from quadrocopters to the payload, and letting each robot calculate its own movement by a mathematical model. 
This paper formulates general conditions for the payload equilibrium but can be scaled to many swarm robots. 
The mathematical model that is defined in the paper is tested with three quadrocopters and offers a stable transportation for objects, with a degree of freedom related to the amount of robots used and can thus be implemented in the field.
Because the swarm robots do not calculate the exact distance between them, this algorithm can be defined as a range-free algorithm. \\

A second algorithm, used for transporting large objects on the ground, in the water and in the air, considers a multi-robot formation control framework, necessary to coordinate the motions of the robots in the group. 
It is mentioned in the paper payload Manipulation Using Cooperative Mobile Multi-Robot Systems, \cite{Mas2012object}, and we will call the algorithm the \emph{Cluster Space Control} algorithm.
This is because control approach used is called cluster space control, and is utilized to control swarm of four four-wheeled robots. 
A user using this type of transportation uses a joystick to input user controls to the whole swarm, after which the control framework specifies the formation and the position and shape of the swarm.
Because the framework calculates the exact range between each robot in the swarm, this algorithm is range-based. 
This control framework allows the user to effectively transport large objects with a scalable set of robots. \\

These two algorithms solve the same problem, namely the problem of transporting large and/or heavy objects, in two entirely different ways. 
The properties are summarized in the following table. 


  \begin{table}[H]
  \renewcommand{\arraystretch}{1.3}
  \caption{User-controlled swarm transportation algorithms}
  \label{table_alg_coltrans1}
  \centering
  \begin{tabular}{|p{3cm}|l|l|l|l|}
    \hline
    \bfseries Algorithm & \bfseries Range &\bfseries Location &  \bfseries Performance & \bfseries Scalability\\
    \hline
     Aerial Equilibrium & Range-free & Location-free & Medium & Low\\\hline
     Cluster Space Control & Range-based & Location-based & Medium & Medium\\\hline

    \end{tabular}
  \end{table}

The main differences between these two algorithms is that in the first mathematical model that is used, each swarm robot individually calculates its position to the other robots and the payload, and is calculating the resulting tension on the payload. 
This way, the rotation and position can be controlled by the position and tension of each robot. 
With the second algorithm, that uses the cluster \emph{cluster space control}, the calculations are not done individually. 
Instead, the framework calculates the position of each robot and the orientation of the payload. 
Thus, these methods both provide stability to the transported payload but in two completely different ways. \\

The problems these algorithms have are that the robots should be coordinated to rely on the location of other robots.
With the mathematical model these are calculated through the tension of the cable attached to each robot, but in the framework it is calculated centrally. 
Problems that are not treated in these articles is how these swarms can locate a payload and autonomously can deliver it to a location. 
Another problem for which no solution is provided is that when these swarms grow to account for heavier objects, the robots will get in each other's way. 
Especially considering transportation, a technique in which every robot must latch on to a payload, this can be hard to practically scale. 

\subsubsection{Autonomous swarm transportation algorithms}
The second sub-poroblem that has to be solved is finding the payload and transporting it to its destination.
This sub-problem can again be divided in four smaller problems:

\begin{itemize}
	\item finding the payload
	\item grabbing the payload
	\item finding target location
	\item moving the payload to the location
\end{itemize}

By defining these sub-problems, the problem is easier to solve and easier to explain each algorithm. \\

Collective transport is the problem in which only one payload has to be moved; we do not concern ourselves with foraging algorithms, which try to find a path to the target to collect multiple objects.
The first algorithm is simple; find the location of the payload through a flocking algorithm and let the swarm robots push the payload to the goal location. 
This algorithm is explained in the paper Collective Transport of Complex Objects by Simple Robots: Theory and Experiments, \cite{Rubenstein2013collecive}, and is called the \emph{flocking} algorithm.
The algorithm is quite simple; every robot looks if it can see the payload location. If so, move towards the location. If not, the robot compares its own heading with the other robots nearby and corrects it until the robot finds the payload location.
Each robot uses one of two behaviors.
 A robot that can see the goals aligns itself towards the goal while moving forward.
 A robot that cannot see the goal uses its IR sensors to determine the heading difference between itself and its neighbors.
 In this paper, the payload location and the goal location are found with light sensors, communicates headings with infrared sensors and uses bump sensors to avoid obstacles. 
The robot latches on to the payload with a velcro strip.
After finding the payload, the robot pushes from an arbitrary side, calculating the amount of force needed in relation to the middle point of mass with its bump sensors and transporting it to the goal location with its light sensor.  
And because the robots latch onto an arbitrary side, scaling of the swarm is extremely effective. \\ 

Another widely used algorithm for transporting objects is the \emph{pheromone} algorithm. 
This algorithms relies on leaving a trail of marks that help guide other swarm robots. 
Although this is algorithm can be used for foraging, it can also be used for retrieving a single payload leading many robots to transport a single payload.
This algorithm is used in the paper Cooperative Transportation by Swarm Robots Using Pheromone Communication. \cite{fujisawa2013cooperative}
 In this paper, this algorithm is tested with swarm-bots which can release and detect spots of ethanol left on the ground, indicating a spot they have to follow. Although the pheromone algorithm implies that pheromones should be used, different varieties of markings can be used, including wireless sensor beacons, drops of paint and even objects. This algorithm is particularly useful for guiding many robots to a payload, if for example the payload is very heavy or if many robots are needed for accurate and/or stable transportation. This algorithm is range-based; it calculates the exact distance to a trail. This algorithm is difficult to implement in dynamic environments, as made marks can dissapear. \\

In looking for swarm algorithms, you want the robots to be as simple as possible. An algorithm that only uses very simple robots is an algorithm that is based on granular convection, als known as the Brazil Nut Effect. This novel method is described in the paper Granular convection to transportation.\cite{sugawara2012object} The algorithms described in this paper are either intended for:

\begin{itemize}
  \item A swarm with homogeneous robots, with no explicit communication
  \item A swarm of heterogeneous robots, composed of robots that change their direction with two different probabilities
  \item A swarm of heterogeneous robots that uses local communication to adjust the fraction of robots that change their direction with higher probability
\end{itemize}

These algorithms listed above are ranked in effectiveness, in which the last one obviously is the most effective, but keep in mind that the robots used to implement this are harder to produce than for the first. \\ The way this algorithm works is as follows. The goal of the swarm robots is to transport the objedct to the goal location. Each swarm robot vibrates with a random force. The goal location outputs a repulsive force. As the robots are randomly placed around the payload and the goal location, the payload gets kicked around by the vibrating swarm robots, until the payload is kicked to the target location. When the swarm increases, the amount of time it costs to transport the payload significantly decreases. With heterogenuous robots, there is more communication between robots to lower the density of the robots around the payload, because slows the process of moving the payload.  This algorithm has different implementations, but is mostly location-free and  range-free. 

A short summarization of these algorithms will be given in the form of this table. 

  \begin{table}[H]
  \renewcommand{\arraystretch}{1.3}
  \label{table_alg_coltrans2}
  \caption{Overview of Collective Transport Algorithms}
  \centering
  \begin{tabular}{|p{5cm}|l|l|l|l|}
    \hline
    \bfseries Algorithm & \bfseries Range & Location & \bfseries Performance & \bfseries Scalability\\
    \hline
    \bfseries Flocking & Range-free & Location-Free & High & High\\\hline
    \bfseries Pheromone & Range-based & Location-Free & Medium & Medium\\\hline
    \bfseries Homogeneous granular\newline convection & Range-free& Location-Free & Low  & High\\\hline
    \bfseries Heterogeneous granular\newline convection & Range-free& Location-Free & Medium & High\\\hline
    \bfseries Heterogeneous granular\newline convection with local\newline communication & Range-based& Location-Free & High & High\\\hline

    \end{tabular}
  \end{table}

These three algorithms are very different from each other, but they all serve a common goal. A problem these algorithms ran into was how to locate the payload, how to move the payload and how to find and reach the goal location. The main problem is locating the payload, something which the granular convection algorithm actually never does.  What these three algorithms have in common is that the transportation process speeds up significantly when adding more swarm robots, but reaches a cap when it becomes ineffective due to maximal density. For this reason, small, easy and cheap to produce swarm robots should be required. The granular convection algorithm has the cheapest robots, but is not necessarily the best, because when the robots become slightly larger and more complicated, more sophisticated and faster algorithms could work. \\

In the mentioned algorithms, moving the payload is mostly done by pushing. In real-life applications this would not always be a good option and a more sophisticated method of attaching should be used. When considering search-and-rescue operations for example, a solution to effectively grab the person to be rescued should be thought of. This is a problem that is yet to be identified in these algorithms and poses a good question for the future.


