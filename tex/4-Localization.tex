%!TEX root = ../Bachelorseminar-RoboticSwarms.tex

Source localization with robotic swarms is an area which has been receiving a lot of research attention in the past few years. 
The main goal is to design an algorithm that effectively allows a swarm of robots to explore an unknown area and find (multiple) source(s).
Source localization can be used for lots of real-world applications.
One of these applications is chemical plume tracing, in which localization is used to detect clouds of high density chemicals. \cite{zarzhitsky2005distributed}
Another example which looks a lot like chemical plume tracing, is radiation source search. \cite{bashyal2008human} 
The difference is that in this case the source of leaking radiation is searched for, and not only for high density clouds. 
A third example is searching for fire, used to assist fire-fighters in their every-day work. \cite{marjovi2009multi}
As can be seen, this technique can be used to look for all kinds of emission sources, if the robots have the right sensors installed. \cite{cui2004swarm}
Although many practical applications can be found, a large amount of the work done in this field is purely theoretical.
This is due to the fact that the price of these individual robots is still rather high and thus it is expensive to produce a swarm.

%Source localization is used in a large category of robotic-swarm applications. 
%Some of the topics include, but are not limited to: chemical plume tracing\cite{zarzhitsky2005distributed}, radiation source search\cite{bashyal2008human}, fire searching\cite{marjovi2009multi} and other types of emission source localization\cite{cui2004swarm}. %A large amount of the work done in this field is purely theoretical, this is due to the fact that the price of these individual robots is still rather high and thus it is expensive to produce a large quantity.
%The main techniques underlying these applications include: control, communication, path-planning and distribution.\cite{Li2012}

\subsection{Algorithms}
Source localization basically exists of two parts: exploring and localization.
In some approaches the robots are randomly distributed over the area, but to do this in real-life applications, some form of exploration or dispersion should be used.
These approaches can be found in section \ref{sec:Exploration} and \ref{sec:Dispersion}.

	\subsubsection{Gradient-based}
		% http://ac.els-cdn.com/S1574119208000518/1-s2.0-S1574119208000518-main.pdf?_tid=fcc58570-aaed-11e3-bbdc-00000aacb350&acdnat=1394742557_d3fd3d52038328cd5bea315cff88955e
		In gradient-based algorithms each robots uses a received signal strength indicator to predict the location of the source with a certain probability, adds this to his map and reports back to a central base.
		This approach is therefore \emph{location-based} and \emph{range-free}.
		This base collects all readings and maps to create a global map and an uncertainty area which is sent back to all robots.
		With this data the robots continuously predict the source position with increasing accuracy as they move towards the source. 
		In common gradient-based algorithms the robots have perfect knowlegde of their location and are limited to single-source searches.
		Compared to random path models it leads to a 40\% reduction in performance time. \cite{zhang2009gradient}
		Furthermore gradient-based algorithms in general perform approximately twice as good as biased random walk algorithm, which will be discussed later. \cite{zhang2009gradient}

	\subsubsection{Particle Swarm Optimization}
		In Particle Swarm Optimization (PSO) a number of particles is randomly distributed over an unkown space of a problem or function. 
		Each particle - in our case a robot - evaluates its current location according to a certain fitness function.
		Then it calculates the best position to go to according to its own historical best position and the historical best positions of the particle(s) in its neighbourhood.
		To prevent the particles from agglomeration a certain randomness is often implemented. 
		By continuously looking for a better position by helping each other, the swarm of particles eventually positions itself at the position of source. \cite{poli2007particle,pugh2007inspiring,derr2009multi}\\
		\\
		In a paper by Jim Pugh and Alcherio Martinoli an attempt to model robotic swarm search based on PSO has been made. \cite{pugh2007inspiring}
		To achieve this a couple of modifications had to be made, for example: changing the normally discrete time to continuous, handling the movement limitations and collisions of robots and limiting the particle neighbourhood of each robot, which is often unlimited in general PSO.
		In this particular paper the algorithm is \emph{location-based} and \emph{range-based}.
		Attempts to make the algorithm \emph{location-free} have been made.
		Conclusions are that either location-based or location-free the performance improves when upscaling the amount of robots.
		When testing the location-free variant, the increase in performance is very abrupt at a certain number of robots while in the location-based variant the performance improves gradually.
		Furthermore detection of the source with small communication range achieves poor results, but improves dramatically as the range increases.\\
		\\
		Another implementation of a decentralized application of PSO is implemented in a paper by Derr. \cite{derr2009multi}
		The developed algorithm is completely \emph{location-free} and \emph{range-based}.
		The targets are equipped with a cell phone that radiates a radio frequency signal that can be detected by the robot, which can wirelessly communicate with limited range.
		The paper shows that a distributed algorithm based on PSO can easily overshoot sources
		This can be prevented with a correction on the standard PSO formula.
		Furthermore it concludes by experiments that the variation in received signal strenghts (RSS) in an indoor environment significantly increases the robot search time in finding the sources.

	\subsubsection{Glowworm Swarm Optimization}
		In Glowworm Swarm Optimization (GSO) the idea is to distribute "glowworms" randomly over the area.
		The glowworms, according to the fitness function, carry a certain lumeniscence quantity called luciferin. 
		The closer they get to the source the more luciferin they contain and the more they attract other glowworms. 
		At every movement step each glowworm moves towards a neighbour within a certain range that carries more luciferin, so they eventually conglomerate at the source(s). 
		The glowworms have a communication range that varies at each step with a certain randomness, to make sure not all robots focus on the same source. \cite{krishnanand2005detection}\\
		\\
		In a paper by Krishnanand a GSO based distributed robotic algorithm is succesfully implemented. \cite{krishnanand2005detection}
		In comparison to PSO based algorithms this algorithm is completely \emph{memoryless}.
		The number of sources found by this particular algorithm is proven to be a strong function of the sensor range.
		Since the number of sources to find is not known in advance, the sensor range is made a varying paramter.
		Besides it shows that when blocking regions the inter-robot communication helps to still let robots select a feasible direction towards the source.

	\subsubsection{Biasing Expansion Swarm Approach}
		In the Biasing Expansion Swarm Approach (BESA) robots have communication possibilities over limited range and together create a ad-hoc wireless network for global communication capibility.
		The algorithm is therefore \emph{location-free} and \emph{range-based}.
		Each robot maintains an occupancy grid map to represent the environment wchich is initiated with all cells unexplored.
		After deployment robots share their locations and sensed concentrations with the swarm.
		To uphold seperation a robot does not enter a cell which is already occupied and maintains cohesion by using a gradual expansion algorithm.
		Robotos can therefore only move to cells that are unexplored, unoccupied and next to another robot in the swarm: the expansion cells.
		To make sure the swarm moves to the emission source, each expansion cell is given a certain biasing parameter, based on the number of robots, the distance between the expansion cell and the robot and the concentration sensed at the particular cell.
		The robot will choose the expansion cell with the highest parameter, so that the swarm will eventually move to the direction of the higher concentrations.
		The BESA algorithm developed in this paper performs twice times better than the general gradient-based approaches. \cite{cui2004swarm}

	\subsubsection{Biased Random Walk}
		A rather simple algorithm inspired by nature is the Biased Random Walk (BSR) approach inspired by bacteria as is described in a paper by Dhariwal \cite{dhariwal2004bacterium}.
		Implementations of this algorithm do not use communication or any form of localization and are therefore \emph{location-free} and neither range-based or range-free.
		Robots only have the possibility to perform two actions: move or tumble. 
		When tumbling the robot just turns into a new random direction. 
		When moving the robot travels a certain distance into the chosen direction. 
		If the robot senses some form of emission from the source, the tumble frequency is lowered, so the distance moved per step increases.
		Eventually the robots will conglomerate at multiple sources.

	\begin{table}[H]
  \renewcommand{\arraystretch}{1.3}
  \caption{Overview of Common Localization Algorithms}
  \label{table_alg_localization}
  \centering
    \begin{tabular}{|l|l|l|l|l|l|}
    \hline
    \bfseries Algorithm & \bfseries Range-type & \bfseries Location-type & \bfseries Performance & \bfseries Scalability\\
    \hline
    \bfseries PSO & Range-based & Location-free & High & High\\
    \hline
    \bfseries GSO & Range-based & Location-free & High & High\\
    \hline
    \bfseries BESA & Range-based & Location-free & Medium & Medium\\
    \hline
    \bfseries Gradient-based & Range-free & Location-based & Medium & Low\\
    \hline
    \bfseries BRW & - & Location-free & Low & High\\
    \hline
    \end{tabular}
  \end{table}

	\subsubsection{Problems}
	The main problem of localization using swarms in the \emph{range-based} approaches is the way robots should cooperate and learn from each other. 
	In other words: how big should the neighbourhood be in which robots cooperate to find sources? 
	One might think it is ideal to use information of all robots.
	However, in case of multiple sources, the best position of all robots can be misleading and cause premature convergence. 
	Therefore in PSO a particle neighbourhood within a certain range or a constant maximum value of neighbours is often chosen. 
	In GSO a range varying each step is implemented to make sure multiple sources can be found.
	Finally in BESA based approaches the robots create an ad-hoc network to create global communication. 

	\subsubsection{Remaining problems}
	There are a lot of factors that affect the performance of localization of multiple sources.
	For example the complexity of the implemented algorithms, the number of robots used or the number of sources that is being searched for. 
	As is mentioned in a paper by McGill however there is a lack of validation cases and reference algorithms that form a ground truth for comparative analysis. \cite{mcgill2011robot} 
	Therefore it is relatively difficult to compare the developed algorithms for (multi-)source localization with robotic swarms. 
	Another question in this field is which approach should be used after a source has been found. 
	Some papers propose to collect the source/target and search further, but this is not always possible.