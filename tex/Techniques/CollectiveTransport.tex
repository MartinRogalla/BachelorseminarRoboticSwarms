%!TEX root = ../../Bachelorseminar-RoboticSwarms.tex

Robotic transportation is an interesting technique. 
Recently Amazon, a large online retailer, announced it would make use of unmanned flying robots to deliver parcels at everyone's door. (\url{www.amazon.com/b?node=8037720011})
But, as is obviously true, each robot can only carry a certain weight, which poses a problem. 
With a robotic swarm of robots it could be possible to transport larger weights in a scalable system, adding swarm robots until you can carry the desired weight. 
Of course, this does not limit itself to aerial robots, but also includes ground robots and even underwater robots. \\
The main problem though of collectively transporting with robotic swarms is to move an object from the start to the destination.
Because the practical applications, like search and rescue operations, are performed on ground, the algorithms for transporting are only applied to ground robots. 
A simple distinction will be made in light of the different algorithms considered: the swarm is completely autonomous or the swarm is controlled by a single ground station.
If a homogeneous swarm is controlled by a single ground station, the algorithm is not considered with finding an object, but is instead focused on safe, accurate and scalable transport. 
In algorithms in which swarms are completely autonomous, the focus is on finding the object and transporting it to its destination.
When considering these two types seperately, both types of algorithms are expanded give a good view of the current state of transporting algorithms.  \\

\subsubsection{User-controlled swarm transportations}
As listed before, first we consider some algorithms concerned with safe and scalable transport of objects, without actually finding the objects. 
Two different types of swarm transport are considered here: ground transport and aerial transport. \\
An algorithm for stable aerial transport is given in the paper Cooperative manipulation and transportation with aerial robots. \cite{Michael2011}. 
In this paper, a scalable solution for stable transport is presented, which can manipulate a payload to a desired pose (position and orientation).
This is done by attaching cables from the flying swarm robots, in this paper quadrocopters are used, to a given object. 
This paper formulates general conditions for the system equilibrium and thus can be scaled for many swarm robots, but is focused on three aerial swarm robots. 
To accomodate for aerial transport, the workspace and the payload stability are very important, and are extensively calculated. 
The mathematical model that is defined in the paper is tested with three quadrocopters and offers a stable transportation for objects, with a degree of freedom related to the amount of robots used and can thus be implemented in the field.
The actual path-finding of the transport from the object to the goal location is not accounted for, as mentioned earlier. \\

A second algorithm, used for transporting large objects on the ground, in the water and in the air, considers a multi-robot formation control framework, necessary to coordinate the motions of the robots in the group. 
It is mentioned in the paper Object Manipulation Using Cooperative Mobile Multi-Robot Systems. \cite{Mas2012}
This particular control approach is called \emph{cluster space control}, and is utilized to control swarm of four four-wheeled robots. 
A user using this type of transportation uses a joystick to input user controls to the whole swarm, after which the control framework specifies the formation and the position and shape of the swarm.
This control framework allows the user to effectively transport large objects with a scalable set of robots.  \\

The main differences between these two algorithms is that in the first mathematical model that is used, each swarm robot individually calculates its position to the other robots and the object, and is calculating the resulting tension on the object. 
This way, the rotation and position can be controlled by the position and tension of each robot. 
With the second algorithm, that uses the cluster \emph{cluster space control}, the calculations are not done individually. 
Instead, the framework calculates the position of each robot and the orientation of the object. 
Thus, these methods both provide stability to the transported object but in two completely different ways. \\

The problems that these algorithms had was that the robots should be coordinated to rely on the location of other robots.
With the mathematical model these are calculated through the tension of the cable attached to each robot, but in the framework it is calculated centrally. 
Problems that are not treated in these articles is how these swarms can locate an object and autonomously can deliver it to a location. 
Another problem for which no solution is provided is that when these swarms grow to account for heavier objects, the robots will get in each other's way. 
Especially considering transportation, a technique in which every robot must latch on to an object, this can be hard to practically scale. 

\subsubsection{Completely autonomous swarm transportations}