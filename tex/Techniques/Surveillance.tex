%!TEX root = ../../Bachelorseminar-RoboticSwarms.tex

Surveillance is the last technique that is going to be discussed in this paper, and for good reason. 
As can be seen in the figure \ref{fig:TechniquesMindMap}, surveillance is a technique that uses both the techniques exploration and localization.
This is obvious, because when a robotic swarm is surveilling an unknown environment, it has to explore the environment and has to localize disturbances.
So it makes sense to review this technique as last, as it references algorithms already used in other techniques.  
Surveillance as a technique is also very close to the application domain, as surveillance is more of a goal than a technique. 
But, because it combines other techniques and is used in many applications, it is deemed useful to review a few popular algorithms. 
In some surveillance algorithms, we also consider the formation technique, in which the robot swarms have to dynamically retain their formation which is important for most of the surveillance algorithms. 

\subsubsection{Surveillance Algorithms}

\textbf{Location-free surveillance}
\begin{itemize}
\item \textbf{Kinetic Theory of Gases Algorithm}\\
The first surveillance algorithm which will be expanded on is a novel one, based on the article Robotic Simulation of Gases for a Surveillance Task. \cite{Kerr2005}.
The goal that is defined for this algorithm is that it requires a homogeneous swarm of robots to monitor a corridor, by sweeping through it while avoiding obstacles. 
The problem reaching this goal is maintaining spatial coverage, especially after passing obstacles, while the swarm robots are only equipped with limited sensors and communication.
The algorithm that is used is described as a Kinetic Theory algorithm, which is modelled after the movement of gases. 
In this algorithm, every molecule of a gas cloud is modeled as a robot, but because robots are not that small, this system is made for a larger scale. 
Every robot has 24 different sonar sensors to detect walls/obstacles, plus the capability to communicate with low bandwidth radio frequency communication. 
The goal direction is detected with a light sensor.\\
The algorithm works as follows. Each robots tries to detect the goal destination. If a robot finds it, it communicates this to all other to specify the direction in which they have to go. 
It is pretty similar to the flocking algorithm, but the difference in algorithms is the way they handle internal and external collisions (i.e. collisions between robots and walls). 
When collisions in the Kinetic Theory algorithm are detected, the robots calculates its new position and orientation depending on the speed and orientation of the obstacle (robot or wall). 
This is modelled after the way particles of a gas cloud move when encountering a virtual collision. 

\end{itemize}

\textbf{Location-based surveillance}
\begin{itemize}
\item \textbf{Networked Robotic Surveillance}\\
	2012 - Ghaffarkhah - Path Planning for Networked Robotic Surveillance
\item \textbf{Surveillance Event Agents}\\
	2006 - Roman-Ballesteros - A Framework for Cooperative Multi-Robot Surveillance Tasks
\item \textbf{Scouts and Rangers}\\
The second surveillance algorithm is also location-based and uses a more straight-forward approach, based on the article A Team of Robotic Agents for Surveillance.  \cite{Rybsky200}
The goal for this algorithm is to explore every dark spot and try to detect motion. 
The algorithm that is used is called the Ranger and Scout behavior, in which the Ranger robot and the Scout robot are heterogeneous robots in the same swarm. 
The algorithm works as follows. The Ranger is a more sophisticated robot than the Scout robot, and keeps track of the environment with an internal grid, where each explored space is represented by a square. 
A square in the virtual grid can be white, gray or black; respectively meaning no obstacles, insufficient knowledge and obstacles. 
The behavior of the Ranger can be described as follows. The ranger will move around, detecting dark spots and obstacles with its sensors.
When a dark spot or an opening in the wall (like a door) has been found, it launches a Scout robot to do reconnaisance, which reports the found information back to the Ranger. 
The RAnger
\end{itemize}

\textbf{Range-free surveillance}
\begin{itemize}
\item \textbf{Stochastic Strategies}\\
	2005 - Grace - stochastic Strategies for Autonomous Robotic Surveillance
\end{itemize}

\textbf{Range-based surveillance}
\begin{itemize}
\item \textbf{Dynamic Directed Movement Behaviour}\\
	2013 - Mullen - Reactive Coordination and Adaptive Lattice Formation in Mobile Robotic surveillance Swarms
\end{itemize}

  \begin{table}[H]
  \renewcommand{\arraystretch}{1.3}
  \caption{Overview of Surveillance Algorithms}
  \label{table_example}
  \centering
    \begin{tabular}{|l|l|l|l|}
    \hline
    \bfseries Algorithm & \bfseries Type & \bfseries Performance & \bfseries Scalability\\
    \hline
    \bfseries Zooi1 & Range-free & High & High\\\hline
    \bfseries zooi2 & Range-free & Medium & Medium\\\hline
    \bfseries Zooi3 & Range-free & Low  & High\\\hline
    \end{tabular}
  \end{table}


\subsubsection{Problems}
There are a lot of problems. Deal with it. 


\subsubsection{Remaining problems}
Yeah, a lot of them. 












