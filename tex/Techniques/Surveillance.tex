%!TEX root = ../../Bachelorseminar-RoboticSwarms.tex

Surveillance is the last technique that is going to be discussed in this paper, and for good reason. 
As can be seen in the figure \ref{fig:TechniquesMindMap}, surveillance is a technique that uses both the techniques exploration and localization.
This is obvious, because when a robotic swarm is surveilling an unknown environment, it has to explore the environment and has to localize disturbances.
So it makes sense to review this technique as last, as it references algorithms already used in other techniques.  
Surveillance as a technique is also very close to the application domain, as surveillance is more of a goal than a technique. 
But, because it combines other techniques and is used in many applications, it is deemed useful to review a few popular algorithms. 
In this surveillance technique, we also consider the formation technique, in which the robot swarms have to dynamically retain their formation, because this is most often combined with surveillance. 

\subsubsection{Surveillance Algorithms}
As listed before, first we consider some algorithms concerned with safe and scalable transport of objects, without actually finding the objects. Because of this, the algorithms listed here are all \emph{location-based}.; the location is assumed to be known. 
Two different types of swarm transport are considered here: ground transport and aerial transport. \\

\textbf{Location-free surveillance}
\begin{itemize}
\item \textbf{Zooi1}\\
\item \textbf{Zooi2}\\
\item \textbf{Zooi3}\\
\end{itemize}

\textbf{Location-based surveillance}
\begin{itemize}
\item \textbf{Zooi1}\\
\item \textbf{Zooi2}\\
\item \textbf{Zooi3}\\
\end{itemize}

\textbf{Range-free surveillance}
\begin{itemize}
\item \textbf{Zooi1}\\
\item \textbf{Zooi2}\\
\item \textbf{Zooi3}\\
\end{itemize}

\textbf{Range-based surveillance}
\begin{itemize}
\item \textbf{Zooi1}\\
\item \textbf{Zooi2}\\
\item \textbf{Zooi3}\\
\end{itemize}

  \begin{table}[!t]
  \renewcommand{\arraystretch}{1.3}
  \label{table_example}
  \centering
    \begin{tabular}{|l|p{2.2cm}|p{2.2cm}|p{2.2cm}|}
    \hline
    \bfseries Algorithm & \bfseries Type & \bfseries Performance & \bfseries Scalability\\
    \hline
    \bfseries Zooi1 & Range-free & High & High\\\hline
    \bfseries zooi2 & Range-free & Medium & Medium\\\hline
    \bfseries Zooi3 & Range-free & Low  & High\\\hline

    \end{tabular}
  \caption{Overview of Surveillance Algorithms}
  \end{table}


\subsubsection{Problems}
There are a lot of problems. Deal with it. 


\subsubsection{Remaining problems}
Yeah, a lot of them. 












