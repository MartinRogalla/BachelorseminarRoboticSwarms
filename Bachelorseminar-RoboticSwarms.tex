\documentclass[a4paper]{article}
%Commented in draft to allow for notes.
\usepackage[top=2in, bottom=1.5in, left=1in, right=1in]{geometry}
\usepackage{float}
\usepackage{hyperref}
\usepackage{tabularx}
\usepackage{tikz}
\usepackage{todonotes}
\usepackage{caption}
\setlength\parindent{0pt}
% \renewcommand{\thefootnote}{\fnsymbol{footnote}}
\usepackage{fancyhdr}
\setlength{\headheight}{15.2pt}
\setlength{\topmargin}{0pt}
\setlength{\textheight}{630pt}
\pagestyle{fancy}

\title{A Brief Survey of Problems and Solutions \\ in Robotic Swarms}
%\title{An Extensive Analysis of Scalability and Performance of Commonly Used Techniques in Robotic Swarms}
\author{S.J.A. Bekhoven  \and
    S.P. Metman \and
    M.J. Rogalla}
\date{\today}

\begin{document}
\maketitle

\begin{abstract}
Swarm robotics has become a prominent and promising research area in the recent years. 
It has great potential use for a large variety of applications. 
In order to not forget the big picture, we believe that a survey of a few basic problems in this field should be given. 
This survey presents a concise overview of a few problems that the robotic swarms research area has faced. For each problem we provide a small discussion as to which approaches were chosen. 
Finally we mention the remaining issues for each problem. 
Each of the algorithms can be categorized by their usage of range information and location information. 
We show the problems which come with each of these categories and compare the scalability and performance of each approach. 
This provides insight into the properties of the algorithms which affect the scalability and the performance. 
Addressing these problems and their approaches gives a better overview and offers inspiration to solve other problems.
Finally we give a general conclusion regarding the recent advancements in robotic swarms.
% As an emerging field, Robotic Swarms has faced many problems. 
% The majority for distributive controlling large networks of robots, commonly referred to as swarms, are either location or range based.
% In this paper we will give a general survey of common problems faced in robotic swarms and the algorithms that solve these problems.
% We have identified the key problems that have been and have not been solved. Finally we make a comparison of the solution, based on performance, scalability and the usage of location information.
\end{abstract}

% Do not compile individual files, compile only this main file.

\section{Introduction}
  %!TEX root = ../Bachelorseminar-RoboticSwarms.tex
\IEEEPARstart{A}{s} robots become smaller and easier to produce, interest for robotic swarms is generated. Many possible applications for robotic swarms exist and it is certain many more will follow, indicated by a growing amount of paper written about robotics at for instance the AAMAS (International Conference on Autonomous Agents and Multi-Agent Systems). \cite{Amigoni2014} Many different applications and techniques exist in the field of robotic swarms. This paper aims to deliver a concise review of these applications and techniques. \\

To avoid confusion, some terminology will be defined. Afterwards, in order to emphasize the importance of the connection between the technology and its applications, a top-down approach is used for this survey. Thus, in this paper there will first be decided on the terminology of the research area, after which some applications will be given. In the second part of the paper we will discuss the most used techniques for these applications and the algorithms behind these techniques. At the end a final overview and a discussion will be given.\\


\section{Formation}
  \label{sec:Formation}
  %!TEX root = ../Bachelorseminar-RoboticSwarms.tex

Formation is the problem of controlling the relative position and orientation of the robots in a group while allowing the group to move as a whole. \cite{consolini2008leader}
Formation is one of the key problems in robotics swarms, as it is a primitive in many other problems and composite-problems.
In particular it is used in the collective transport problem, where a swarm has to hold a formation to move an object. 
An example of such an application is moving an object with robotic cars. \cite{mas2012object}
An application in which formation is also extensively used is area surveillance, where formation is used to increase coverage. \cite{burkle2011towards}
A more specific example is indoor surveillance, where ground robots are already being used. \cite{di2010autonomous, rybski2000team}
\\

Two main subproblems arise from the original formation problem: the \emph{communication problem} and the \emph{formation stability} problem.\\
In the \emph{communication subproblem} the algorithm needs to come with a communication algorithm for usage within the robotic swarm. 
Each of these algorithms differ in these inter-robot communication strategies. 
Some algorithms only rely on local communication while other algorithms communicate via a centralized system.
This affects if an algorithm is range-free or range-based. 
Some algorithms keep track of each robot's location and are thus location-based, while others are location-free.
These locations can be be communicated through different ways of communication. \\

The \emph{formation stability problem} is another subproblem, which each of these algorithms have to deal with.
Specifically, the algorithm should be able to dampen the effects of disturbance propagation. 
This means that when one robot in the formation encounters a disturbance and moves out of formation, how is this situation handled in the swarm. 
The effect of such a disturbance should be reduced.
We call this the \emph{source disturbance dampening} subproblem. 
For each algorithm, we review how these problems are solved.

\subsection{Algorithms}
Many different algorithms have been used to solve the formation problem. \cite{chen2005formation,consolini2008leader}
These algorithms can categorized as: leader-follower algorithm \cite{consolini2008leader,das2002vision}, 
behavior-based algorithm \cite{balch1998behavior,lawton2003decentralized}, 
and virtual structure algorithm \cite{ren2004decentralized,do2007nonlinear}. \\
The algorithms are build upon in later algorithms to create more novel algorithms. A few of these algorithms are: 
virtual space configuration \cite{wee2013formation}, 
fuzzy formation control \cite{ranjbar2012novel},
and team-work software control. \cite{kaminka2013use} 
We compare these algorithms primarily by their performance and their scalability. 
We define \emph{performance} in how quickly the formation can be formed again after the formation is lost (disturbance dampening). 
This can also be referred to as the stability of the formation.
We discuss each of these algorithms separately in the next section and make a comparison at the end of the chapter.

\subsubsection{Leader-follower algorithm}
First, we discuss the leader-follower algorithm. 
In the leader-follower algorithm a robot of the swarm is designed as the leader.
The leader moves along a predefined path while the other robots, the followers, are maintaining a desired distance and orientation to the leader. \cite{consolini2008leader}
This can be implemented by equipping the leader robot with a omni-directional camera and logical sensors, like an obstacle detector and a collision detector.
The leader then instructs the followers through local communication. \cite{das2002vision}
The main problem with this algorithms is that it depends heavily on the leader and when something happens to the leader, the algorithm fails. 
Source disturbance dampening is done by the leader; the leader communicates any disturbances to its followers.

\subsubsection{Behavior-based algorithm}
The second algorithm we discuss is the behavior-based algorithm. 
In the behavior-based algorithm, every robot in the swarm is programmed with a certain behavior. 
These behaviors may differ between robots.
Some examples of these behaviors are the collision-avoidance behavior and the target-seeking behavior. 
The action that is taken is decided by weighing the relative importance of each behavior. \cite{consolini2008leader}
These behaviors can then be used to maintain certain formations like a line, a column, a diamond and a wedge. \cite{balch1998behavior}
The dynamics and stability of this algorithm are calculated with the Lapyunov function, which is used to account for many stability issues. \cite{lawton2003decentralized}
In this algorithm, the decisions are not made locally, but real-time data is sent to a system which then decides what each robot should do based on their behaviors.
Disturbance source dampening is applied by the different behaviors assigned to each swarm robot.

\subsubsection{Virtual structure algorithm}
The last of the more basic algorithms is the virtual structure algorithm. 
The virtual structure algorithm considers the formation as a single virtual rigid structure such that the behavior of the robotic system is similar to that of a physical object. 
Desired trajectories are not assigned to each single robot but to the entire formation as a whole. 
The behavior of the formation in this case is exactly predictable but also generates a  large overhead. \cite{consolini2008leader}
Such a virtual structure is thus decentralized and appropriate distributed control strategies can be made. \cite{ren2004decentralized}
The control system can be structured in four sequential stages: defining the dynamics of the virtual structure, 
translating the motion of the virtual structure into the desired motion for each robot, 
deriving output feedback for each robot,
and introducing formation feedback from each robot to the virtual structure. \cite{do2007nonlinear} 

\subsubsection{Virtual space configuration}
Aside from these three algorithms which have become a standard way for implementing formation control, more novel algorithms have been developed. 
One of these algorithms is the virtual space configuration algorithm. \cite{wee2013formation}
This algorithm uses a leader-follower algorithm, but each robot uses virtual space and virtual robots inside that space.
These virtual robots are then used by each robot to avoid collisions. 
The main difference is then that multiple follower robots in formations can maintain formation without crashing (using only virtual collisions), and they can have scalability of formation. 
This algorithm is range-based and location-free, but produces a lot of overhead in local communication.

\subsubsection{Fuzzy formation control}
Another novel algorithm is fuzzy formation control. \cite{ranjbar2012novel}
In robotic swarms, optimal control techniques have been found to treat robust stabilization and tracking problems. 
But in these cases, the exact model of the system has to be known. 
But this fuzzy formation control technique provides a robust scheme with a fuzzy logic estimator to implement effective controls for uncertain dynamic models.
This way, the exact model does not need to be known, as nonlinearity, external disturbances and model uncertainties are dealt with by fuzzy approximation.

\subsubsection{Teamwork software control}
Lastly, the last formation control we discuss is the teamwork software control. \cite{kaminka2013use}
This algorithm combines different control algorithms and tries to integrate this in one formation control algorithm.
It is mostly behavior-based, and each of the robots in the swarm has its own behavioral process. 
But instead of carrying out its behavior directly, each robot communicates with each other to set a ``team flag''. 
When every robot agrees with this flag, only then is this order carried out. 
This resembles the virtual structure control algorithm a little, as in that the swarm is considered as a whole. 
This algorithm is location-free but range-based, and produces more overhead than the behavior-based algorithm.

All the algorithms try to solve the formation problem. 
%The algorithms are all location-free, due to the fact that they have to work with relative location and not absolute locations. If there is
These algorithms are all location-free, which makes sense.  
Because when an algorithm is location-based, all robots know their exact locations and can adjust their position, and the problem is already solved. \\
The most important attribute for each algorithm is the performance, so how stable the formation is, as in how good the formation can be kept. 
This performance increases in the later algorithms, which expand upon the more basic algorithms. 
But, as the performance increases, the scalability lower for algorithms which communicate via a central framework. 
Take for example the teamwork software control algorithm. 
This algorithm provides high performance, but also creates lots of overhead activity due to its central communication.
This decreases the scalability of this algorithm significantly.

\subsection{Discussion}
Concluding this chapter we discuss the future problems of these algorithms. 
As can be seen from the algorithms, centralized communication reduces scalability. 
Although centralized communication does provide high performance, the lower scalability is a downside. 
Decentralized algorithms are more likely to be scalable, and is an interesting topic for future solutions to the formation problem.
Another problem faced by these algorithms is reliability, especially for the leader-follower algorithm. 
If the leader is lost (for example destroyed), the algorithm does not work anymore. 
A new leader could be chosen among the followers, but this is not taken into account in all the leader-follower algorithms. 
This is a problem that most of these algorithms do not take in account.
  \begin{table}[H]
  \renewcommand{\arraystretch}{1.3}
  \label{table_alg_formation}
  \centering
\scalebox{0.85}{
  \begin{tabular}{|l|l|l|l|l|}
    \hline
    \bfseries Algorithm & \bfseries Range & \bfseries Location & \bfseries Performance & \bfseries Scalability\\
    \hline
    Leader-follower & Range-based & Location-Free & Medium & Low\\\hline
    Behavior-based & Range-free & Location-Free & Medium & High\\\hline
    Virtual structure& Range-based& Location-Free & High  & Medium\\\hline
    Virtual space & Range-based& Location-Free & High  & Low\\\hline
    Fuzzy control& Range-free& Location-Free & High  & Medium\\\hline
    Teamwork control& Range-based& Location-Free & High  & Low\\\hline
    \end{tabular}
    }
  \caption{Overview of Formation Algorithms}
  \end{table}


\section{Dispersion}
  \label{sec:Dispersion}
  %!TEX root = ../Bachelorseminar-RoboticSwarms.tex

We consider dispersion as one of the key problems in robotic swarms. \cite{ugur2007dispersion,mclurkin2007distributed,ludwig2006robotic}
Dispersion can be compared to the blanket coverage problem, which is defined in one of the first papers written on the topic of swarm robots: \emph{Command Control for Many-Robot Systems}. \cite{gage1992command}
The objective of blanket coverage is defined as ``achieving a static arrangement of elements that maximizes the detection rate of targets appearing within the coverage area.'' \cite{gage1992command}
The difference between dispersion and formation is often confused.
While formation is trying to maintain explicitly specified spacing relationships, dispersion tries to find the best ``formation'' for the current environment.\\
Dispersion has a great number of applications.
Some of these applications include, but are not limited to exploration, surveillance, military response, disaster response and planetary exploration. \cite{ludwig2006robotic,Penders2011,mclurkin2007distributed} 

\subsection{Algorithms}
In this section we discuss a few categories of approaches which are used to solve the dispersion problem.
First we discuss the simpler algorithms which are based on the approach which includes randomness such as random-walk and wall-following.
After that we show which problems the dispersion problem has faced in the past and how the newer algorithms have solved these problems.
Further we motivate as to why the algorithms are put into a specific category and discuss the scalability and performance for each type of algorithm.
When discussing the \emph{performance} we mean the time and ability to fully create a uniform dispersion in any given environment, whether it is static or dynamic. When discussing the \emph{scalability}, we mean the ability of maintaining the same performance when the population in the robot swarm is increased.

\subsubsection{Random-based Approach}
The algorithms in this category solve the dispersion problem with an approach which relies on randomness to perform the dispersion.
All of these algorithms are mainly location-free and primarily do not measure distances making them also range-free.\\
The most pre-eminent algorithm in this category is the \emph{Random-Walk} algorithm due to its simplistic nature and efficiency. \cite{morlok2007dispersing}
The \emph{Random-Walk} algorithm changes the robot's orientation randomly and moves it forward until an obstacle is detected.
The algorithm then repeats the steps indefinitely.
The algorithm is very scalable, due to the simple set of steps that have to be performed and due to its non-interdependence of other robots in the swarm.
A minor drawback with the \emph{Random-Walk} algorithm is that it does not guarantee uniform dispersion.
Due to the reliance on randomness, the algorithm is not optimal.
If low-energy consumption has a high priority in the system, we do not recommend the usage of this algorithm.
In all other situations which allow the usage of the \emph{Random-Walk} algorithm, the algorithm is greatly recommended due to its satisfying performance and highly rated scalability.\\
Another algorithm which is also used for dispersion purposes is the \emph{Follow-Wall} algorithm. \cite{morlok2007dispersing}
Originally the algorithm is not random, but follows an exact set of rules depending on the environment.
The algorithm is created for a non-dynamic environment and not for usage in robotic swarms.
The robot is unable to see the difference between robots and other obstacles. This creates the possibility for robots to constantly follow each other, while thinking that they are actually moving around a wall or other static obstacles.
The performance rating is low due to the meager amount of randomness and its inability to guarantee the uniformity of the dispersion.
The scalability is also rated low, due to the performance deterioration when increasing the scale.

\subsubsection{Graph Theory Approach}
The most-common dispersion algorithms which use a graph theory approach can be divided into two categories: tree-search algorithms and graph-connectivity algorithms.\\
The tree-search inspired algorithms have two strategies: the \emph{Depth-First Leader-Follower(DFLF)} Strategy and the \emph{Breadth-First Leader-Follower} Strategy. \cite{hsiang2004algorithms}
These strategies rely on building a tree of all the possible moves that each robot can can make.
This tree can be created if and only if it is possible to have a total overview of the environment in which the bots are located.
Therefore the implementation of these algorithms are mainly considered as \emph{location-based} algorithms.
The \emph{BFLF} algorithm requires the robots to travel less and has reduced complexity level compared to the \emph{DFLF} algorithm.
Still it maintains the same quality of the solution and thus has our preference.
The performance of these algorithms is great in terms of quality of the solution.
However, due to the high complexity of these types of algorithms, the scalability of this type of algorithms is rather low.
Another downside is that the centralized control method brings is a much greater risk of failure in hostile environments.\\
One of the algorithms regarding connectivity in graphs is the clique-intensity algorithm. \cite{ugur2007dispersion}
The clique-intensity algorithm is range-free since it simply tries to detect other surrounding swarm-bots within the limited connection range and does not have the need to measure distances.
The performance and scalability of the algorithm is very high, due to decentralized control.
The clique-intensity algorithm faces problems due to the fact that there are high amounts of noise in the wireless intensity signals when used in real-world applications.

\subsubsection{Artificial Potential Fields Approach}
Artificial Potential Fields (APF) algorithms use fields to repel robots away from each other and attract them to their goal. \cite{khatib1986real}
There are many differences between implementations of these algorithms, but all implementations use relative locations, both measuring and bearing.
Therefore the algorithms which use this approach are location-free. \cite{pakanati2010swarm}
The algorithms are well-scalable due to their relatively low complexity.
The performance of the algorithms is rated high, as the robots are uniformly dispersed.
Some minor problems with this type of algorithm exist.
Minor changes in sensing positions from other robots can cause ripple effects throughout the network.
This can however be solved by incorporating \emph{dead-zones}.
These dead-zones allow very minor changes in location, without the immediate reaction of the network and thus the ripple effects are reduced. \cite{pakanati2010swarm}

\subsubsection{Inverse-Vector Approach}
Some examples of the Inverse-Vector approach are the seek-open algorithm \cite{morlok2007dispersing}, the Fiducial algorithm \cite{morlok2007dispersing} and the Uniform Directed Dispersion (UDD) algorithm \cite{mclurkin2007distributed}.
Each of the algorithms sense where obstacles and other robots are relatively positioned and calculate a vector of that data.
Afterwards they calculate the inverse of that vector and move into that direction.
The \emph{Fiducial} algorithm has an advantage over the \emph{Seek-Open} algorithm in that it uses a beacon like system.
This prevents robots from running into each other and encourages uniformity of the distribution.
The seek-open and UDD algorithm use other distance measures mostly using ultrasonic sensors.
In both cases the algorithm is a range-based algorithm, where every robot is able to get the relative location.
It is important to stress that the algorithms should perform periodic checks to detect dynamic changes in the environment, such as other moving robots.
\subsection{Discussion}
There are many problems with the implementation of the dispersion algorithms.
Many problems have been solved, but there are also some remaining problems left to be solved.
In this section we discuss both topics.
Some problems concerning small fluctuations in relative position data can have a major effect in range-based approaches such as the Artificial Potential-Field approach and the Inverse-Vector approach.
This has been solved in \emph{Swarm Dispersion via Potential Fields, Leader Election, and Counting Hops} with the introduction of \emph{dead-zones}.
These dead-zones are limited error ranges, which allow for small location changes, without affecting the whole robotic swarm.
In location-based algorithms, there are problems with the building of the exact overview of the environment.
Very often applications do not allow for this, due to for example dynamic changes in the environment or a lack of observability.
Research in this area should be focused on high-accuracy real-time observation tools in combination with range-based algorithms which prevent obstacle-collision.

  \begin{table}[H]
  \renewcommand{\arraystretch}{1.3}
  \label{table_alg_dispersion}
  \centering
\scalebox{0.85}{
    \begin{tabular}{|l|l|l|l|l|l|}
    \hline
    \bfseries Algorithm & \bfseries Approach & \bfseries Range & \bfseries Location & \bfseries Performance & \bfseries Scalability\\
    \hline
    Random Walk & Random-Based & Range-Free & Location-Free & Medium & High\\\hline
    Follow Wall & Random-Based & Range-Free & Location-Free & Low & Low\\\hline
    DFLF & Graph-Theory & Range-Based & Location-Based & Medium-High & Low\\\hline
    BFLF & Graph-Theory & Range-Based & Location-Based & High & Low\\\hline
    Clique-Intensity & Graph-Theory & Range-Free & Location-Free & Medium & High\\\hline
    APF & Potential-Fields & Range-Based & Location-Free & High & High\\\hline
    Directed Dispersion & Inverse-Vector & Range-Based & Location-Free & Medium & High\\\hline
    Seek Open & Inverse-Vector & Range-Based & Location-Free & Low & Medium\\\hline
    Fiducial & Inverse-Vector & Range-Based & Location-Free & Medium & High\\\hline
    \end{tabular}
    }
  \caption{Overview of Common Dispersion Algorithms}
  \end{table}


\section{Exploration}
  \label{sec:Exploration}
  %!TEX root = ../Bachelorseminar-RoboticSwarms.tex

Another fundamental problem in swarm robotics is the task to fully explore an unknown environment.
The main goal is to minimize the overall exploration time while still exploring the whole environment. 
The main problem faced when trying to achieve this goal is finding appropriate target points for each individual robot so that they simultaneously explore different regions of the environment. \cite{burgard2005coordinated} \\
Exploration is found in many robotic swarms problems, for example in \emph{path-finding}, \emph{collective transport} and \emph{surveillance}.
Practical applications that use exploration are for example rescue missions. \cite{Naghsh2008,Penders2011}
In this particular paper, a robotic swarm applying exploration is used to assist navigation for firefighters.
It is used in situations in which their vision is blocked by smoke and obstacles. 
A last example of an application is cleaning. \cite{wagner2008cooperative}
Here, exploration is used to clean a surface with cleaning robots as fast and as efficient as possible. 
Exploration is used in many more different robotic swarm applications and is a building block for many other problems.

%OLD
%Exploring an environment is one of the fundamental problems faced in mobile robotics. 
%The main goal is to minimize the overall exploration time and the main problem faced when trying to achieve this goal is finding appropriate target points for each individual robot so that they simultaneously explore different regions of the area \cite{burgard2005coordinated}. 
%Robotic swarm exploration can be used for real-world applications like rescue missions \cite{Naghsh2008,Penders2011}, surveillance \cite{Burkle2010} and cleaning \cite{wagner2008cooperative}.
 
%!TEX root = ../../main.tex

\subsection{Algorithms}
The exploration problem has been studied in detail for single robots \cite{lee1997quantitative,albers1999exploring} as well as for robotic swarms.
The first real approach towards robotic swarm exploration is based on finding frontiers, cells that are reachable and adjacent to unexplored cells. \cite{yamauchi1998frontier} This approach is extended in multiple ways of which we will mention two. \cite{solanas2004coordinated,sheng2006distributed}
After that, we will discuss some other solutions to the exploration problem. \cite{singh1993map,zlot2002multi}
We will also discuss three solutions to an extension of the exploration problem which is called foraging. \cite{jung2010multi,hoff2010two}
Foraging basically comes down to exploring an unknown area, finding the optimal path to a certain target and bringing that target back to the base.
Since in this paper we will only handle source localization and foraging has a lot overlap with exploration, we will discuss it in this section.
When comparing we define the \emph{performance} of the exploration algorithms as the explored area per distance traveled.

	\subsubsection{Frontier-based Algorithms}
	A first approach is frontier-based exploration, which is inspired by the question ``Given what you know about the world, where should you move to gain as much new information as possible?''.
	Its goal is to gain as much information as possible when traveling towards a new location.
	An evidence grid is used in which the occupancy probability is stored for each cell, so the algorithm is \emph{location-based}.
	The cells have, dependent to their occupancy probability, a certain state which is either open, unknown or occupied.
	Each open cell that is adjacent to an unknown cell is labeled as a frontier cell.
	Every group of frontier cells above a certain size is considered a frontier.
	Once the frontiers have been detected, the robot navigates towards the nearest unvisited frontier.
	If the robot is unable to make progress to its destination, it will add the frontier to the list of inaccessible frontiers.
	Each time a robot does or does not reach a frontier it performs a sensor sweep and adds the new information to a local grid, which is communicated and then merged with every robots global map. \cite{yamauchi1998frontier}\\
	\\
	A limitation of the general frontier-based approach is that since navigation is independent, robots may waste time navigating to the same frontier.
	This will either cause an avoidance maneuver or the robots will block each other.
	In the last case the robots will mark their destination frontiers as inaccessible which of course is not preferable.
	Furthermore global communication is assumed, which in real-world applications is  impossible since robots have limited communication range. 
	Several attempts have been made to solve the problems described at the previous section by for example implementing a bidding algorithm based on the frontier principle.
	The robots first select a certain target according to the frontier-based algorithm.
	It then broadcasts a bid according to: the target of the robots in the sensor range, the distance to the targeted cell and a nearness measure.
	This nearness measure is a factor which is dependent of the number of robots in the neighborhood to keep the robots together and thus sustain communication.
	After waiting for constant time, if the robot bids the highest value, it travels towards its target.
	This approach makes sure two robots never travel towards the same frontier and tend to stay together, so communication stays possible.
	This approach results in better communication, less repeated coverage and less exploration time. \cite{sheng2006distributed}\\
	\\
	Another limitation of the general frontier-based approach is that it is possible that the robots concentrate in certain parts of the area and therefore reach other parts much later.
	In case of for example search and rescue missions, it is important that the area explored more gradually to spot targets in every part in short time. 
	An approach that solves this problem is the algorithm based on K-means clustering.
	First the algorithm divides the unknown space in as many regions as robots by the K-means clustering algorithm with $K$ the number of robots.
	The algorithm then assigns all robots to a certain region by distance calculation, making a difference between accessible and inaccessible regions.
	After that each robot gets assigned to a frontier cell in its region.
	Robots with accessible regions will be assigned to frontier cells by the distance between the robot and the frontier cell and a penalty when the cell is close to another frontier cell assigned to a different robot.
	Robots with inaccessible regions get another penalty for the distance from the frontier cell and the centroid of their own region.
	When this formula gets maximized, all robots will tend to choose frontiers in their own region or close to their own region.
	If a robot reaches its destination, the target assignment will be repeated. \cite{solanas2004coordinated}

	\subsubsection{Market Economy Algorithm}
	A completely different approach is based on a market economy.
	Robots generate a list of goal points via a simple strategy, for example randomly or greedily.
	The robot then sets up an auction to sell its goal points and tries to buy better ones from other robots.
	Each goal point is awarded a certain revenue according to the amount of information it will provide and a certain cost according to the resources it will use to achieve it.
	When they have tried to sell all of their tasks and have bought the interesting ones, the robot orders its goal point list greedily on distance and sets course for the first one in the list.
	At regular intervals the robots exchange pieces of their map to each other for a certain cost/revenue depending on the expected utility.
	The relatively simple algorithms used to generate goal points should be optimized by maximizing benefit (information gained) while minimizing the costs (in terms of travel distance).
	By allowing the robots to communicate via the market place architecture, the performance increases with a factor of $3.4$ compared to a random walk in a four robot system. \cite{zlot2002multi}

	\subsubsection{Heterogeneous Exploration Algorithm}
	In this approach robots with different size, speed and sensor range are used.
	The unknown area is treated as an occupancy grid with a status for each cell: occupied or free.
	All robots start by filling a space quantum varying in size according to the specification of the robots.
	If they explore a cell in the occupancy grid, the status is set to occupied and is from then on seen as an obstacle by all robots.
	When the robot sees a cell it cannot reach, it has found what we call a \emph{tunnel}.
	The robot than places information about the tunnel at the robot call queue of another smaller robot.
	If that robot is not able to reach the unexplored cell as well, it will pass it to an even smaller robot and so on.
	When a robot finishes exploring its space quantum it either travels towards the tunnel placed on its queue or starts exploring the next adjacent space quantum.
	The robots continuously share their occupancy grid and the motion they are intending to avoid collisions.
	This approach assumes global and very intensive communication amongst all robots which is very hard in real-life applications. \cite{singh1993map}

	\subsubsection{Foraging-based Exploration}
	Foraging algorithms are based on the foraging behavior of ants to find food sources. \cite{hoff2010two}
	For finding these food sources, they have to explore an area first. 
	So, foraging algorithms are essentially extensions of exploration algorithms. 
	The difference between the two is that foraging algorithms also have to plan an optimal path to target locations. \\	
	Based on the foraging behavior of ants, two different algorithms can be used: the Virtual Pheromone algorithm and the Cardinality algorithm.
	Ants can leave marks in the form chemical pheromones behind for other ants, which they use to decide which way they go. 
	The Virtual Pheromone algorithm uses two different types marks instead of many, to keep the robots low-level. 
	Some robots in the swarm are used as beacons, others scout the area for food sources. 
	For the same reason, the marks are made virtual by direct local communication between robots in the swarm.
	So, the algorithm is range-based. 
	It is location-free because the robots do not keep track of their absolute location.\\
	The second foraging algorithm is the Cardinality algorithm. 
	This works similar to the Virtual Pheromone algorithm, as in that walker robots can become beacons. 
	But in this algorithm the beacons send out its cardinality: the number of beacons between that beacon and the nest. 
	This algorithm is range-based and location-free too.
	Conclusive results are produced in the article that the Cardinality-algorithm outperforms the Virtual Pheromone algorithm, proving that the ant pheromone algorithm can be improved.\\

	A different foraging-based exploration algorithm is the Multihop Communication algorithm. \cite{jung2010multi}
	The problem that is defined here is how a swarm of robots can find its way out of a random maze as fast as possible.
	To do this, every robot in the swarm explores the maze, and generates a map through its sensor input.
	Then, it communicates this to every other robot using a technique similar multihop communication found in computer networks.
	Eventually, a map of the whole maze is generated out of which the shortest path to the exit can be collectively calculated.
	This algorithm is range-based, because the robots communicate locally when close in proximity.
	It is also location-based, because they try to map their absolute location in the maze. 


	
\subsection{Discussion}


	- coordinated vs collaborative
	- global communication, real-life applications
	- some work has been done

	- trade-off between cost and revenue
	- how long to travel to get new information


\begin{table}[H]
	\renewcommand{\arraystretch}{1.3}
	\label{table_alg_exploration}
	\centering
  \scalebox{0.85}{
    \begin{tabular}{|l|l|l|l|l|l|}
	    \hline
	    \bfseries Algorithm & \bfseries Range-type & \bfseries Location-type & \bfseries Performance & \bfseries Scalability\\
	    \hline
	    Frontier-based & Range-free & Location-based & Medium & Low\\
	    \hline
	    Frontier-based with bidding & Range-based & Location-based & Medium-high & High\\
	    \hline
	    Frontier-based with K-means clustering & Range-free & Location-based & High & High\\
	    \hline
	    Market Economy & Range-based & Location-based & High & Medium\\
	    \hline
	    Heterogeneous Exploration & Range-free & Location-based & Medium & High\\
	    \hline
	    Virtual Pheromone & Range-based & Location-free & Medium & High\\
	    \hline
	    Cardinality & Range-based & Location-free & High & High\\
	    \hline
	    Multihop Communication & Range-based & Location-based & Medium & Medium\\		
	    \hline
    \end{tabular}
  	}
	\caption{Overview of Common Exploration Algorithms}
\end{table}

	% 

	%\subsubsection{Market-economy based}
	%Zlot et al. have chosen another approach which also uses some kind of bidding principle called a market architecture. 
	%The robots have perfect knowledge of their location and keep track of a map.
	%Furthermore the robots communicate with each other over limited range, so this is an \emph{location-based} and \emph{range-based} approach.
	%The distributed algorithm starts by generating a list of goal points.
	%Goal points are generated to navigate to with three different strategies.
	%The strategies used by the robots may be all the same, vary across robots or even over time.
	%The three strategies are 1) random, 2) greedy exploration and 3) space division by quadtree.
	%All strategies are very simplistic, because the intention is that the market architecture removes the inefficiencies.
	%The goal points generated are greedily ordered (shortest-path).
	%The robot then tries to sell all of its tasks to robots it can communicate with by auction for a specified amount of time.
	%Each goal point is awarded a certain revenue according to the amount of %information it will provide and a certain cost according to the resources it will cost to achieve it.
	%	The robots start bidding and the highest bidder is awarded the task if it is higher than the minimum amount set by the auctioneer.
	%	After trying to sell all of its tasks by auction the robot travels towards a goal point.
	%	If it reaches the goal point, it generates a certain amount of new goal points, starts off with its next goal and offers its remaining goals to other robots.
	%	Finally at regular intervals robots can exchange pieces of their own map with each other for a certain cost/revenue depending on the expected utility.
	%	When asked, all robots send their maps to a central base, so a global map can be created.\\
	%\\
	%Surprisingly simulation shows that when only using the random strategy the algorithm performs just as good as when using the quadtree strategy.
	%Furthermore the algorithm using the random or quadtree strategy is compared to a communicationless situation.
	%We can see that allowing the robots to communicate via the market place architecture improves the exploration efficiency with a factor of $3.4$ in a four-robot system.
	%The system could be improved by for example using a time-based cost scale instead of a distance-based for minimizing while exploring.




%Coordinated multi-robot exploration
% http://ieeexplore.ieee.org/stamp/stamp.jsp?tp=&arnumber=1435481
%\cite{burgard2005coordinated}

%Collaborative multi-robot exploration
% http://www.cs.cmu.edu/afs/.cs.cmu.edu/Web/People/motionplanning/papers/sbp_papers/integrated2/burgard_multi_robot_explor.pdf
%\cite{burgard2000collaborative}

%Coordination for Multi-Robot Exploration and Mapping
% http://isl.ecst.csuchico.edu/DOCS/Papers/simmons2000coordination4MultirobotExploration.pdf
%\cite{simmons2000coordination}

%Multi-robot exploration Controlled by a Market Economy\\
%Market-place algorithm, seems good, ++
% http://repository.cmu.edu/cgi/viewcontent.cgi?article=1174&context=robotics&sei-redir=1&referer=http%3A%2F%2Fscholar.google.nl%2Fscholar%3Fq%3Dmulti-robot%2Bexploration%26btnG%3D%26hl%3Dnl%26as_sdt%3D0%252C5#search=%22multi-robot%20exploration%22
%\cite{zlot2002multi}

%Multi-robot exploration under the constraints of wireless networking
%http://ac.els-cdn.com/S0967066106001547/1-s2.0-S0967066106001547-main.pdf?_tid=2dbdf35e-ab00-11e3-b5b9-00000aab0f6c&acdnat=1394750480_8d4547f2c72259f166f10342891ca745
%\cite{rooker2007multi}

%A practical, decision-theoretic approach to multi-robot mapping and exploration
% http://ieeexplore.ieee.org/stamp/stamp.jsp?tp=&arnumber=1249654
%\cite{ko2003practical}

%Multi-robot collaboration for robust exploration
%\cite{rekleitis2001multi}
% http://download.springer.com/static/pdf/926/art%253A10.1023%252FA%253A1016636024246.pdf?auth66=1394923400_27944ba6f349eb21a47a2c60f843ab9f&ext=.pdf

%Distributed multi-robot coordination in area exploration +++++
%\cite{sheng2006distributed}
% http://ac.els-cdn.com/S092188900600114X/1-s2.0-S092188900600114X-main.pdf?_tid=33bf4874-ab01-11e3-a577-00000aacb35f&acdnat=1394750920_0424c8c4f3098477e3985d83f4f339f2

%Coverage for robotics–A survey of recent results

%multi-robot coverage??

 
\section{Source Localization}
  \label{sec:Localization}
  %!TEX root = ../Bachelorseminar-RoboticSwarms.tex

Source localization with robotic swarms is an area which has been receiving a lot of research attention in the past few years. 
The main goal is to design an algorithm that effectively allows a swarm of robots to explore an unknown area and find (multiple) source(s).
Source localization can be used for lots of real-world applications.
One of these applications is chemical plume tracing, in which localization is used to detect clouds of high density chemicals. \cite{zarzhitsky2005distributed}
Another example which looks a lot like chemical plume tracing, is radiation source search. \cite{bashyal2008human} 
The difference is that in this case the source of leaking radiation is searched for, and not only for high density clouds. 
A third example is searching for fire, used to assist fire-fighters in their every-day work. \cite{marjovi2009multi}
As can be seen, this technique can be used to look for all kinds of emission sources, if the robots have the right sensors installed. \cite{cui2004swarm}
Although many practical applications can be found, a large amount of the work done in this field is purely theoretical.
This is due to the fact that the price of these individual robots is still rather high and thus it is expensive to produce a swarm.

%Source localization is used in a large category of robotic-swarm applications. 
%Some of the topics include, but are not limited to: chemical plume tracing\cite{zarzhitsky2005distributed}, radiation source search\cite{bashyal2008human}, fire searching\cite{marjovi2009multi} and other types of emission source localization\cite{cui2004swarm}. %A large amount of the work done in this field is purely theoretical, this is due to the fact that the price of these individual robots is still rather high and thus it is expensive to produce a large quantity.
%The main techniques underlying these applications include: control, communication, path-planning and distribution.\cite{Li2012}

\subsection{Algorithms}
Source localization basically exists of two parts: exploring and localization.
In some approaches the robots are randomly distributed over the area, but to do this in real-life applications, some form of exploration or dispersion should be used.
These approaches can be found in section \label{sec:Exploration} and \label{sec:Dispersion}.

	\subsubsection{Gradient-based}
		% http://ac.els-cdn.com/S1574119208000518/1-s2.0-S1574119208000518-main.pdf?_tid=fcc58570-aaed-11e3-bbdc-00000aacb350&acdnat=1394742557_d3fd3d52038328cd5bea315cff88955e
		In gradient-based algorithms each robots uses a received signal strength indicator to predict the location of the source with a certain probability, adds this to his map and reports back to a central base.
		This approach is therefore \emph{location-based} and \emph{range-free}.
		This base collects all readings and maps to create a global map and an uncertainty area which is sent back to all robots.
		With this data the robots continuously predict the source position with increasing accuracy as they move towards the source. 
		In common gradient-based algorithms the robots have perfect knowlegde of their location and are limited to single-source searches.
		Compared to random path models it leads to a 40\% reduction in performance time. \cite{zhang2009gradient}
		Furthermore gradient-based algorithms in general perform approximately twice as good as biased random walk algorithm, which will be discussed later. \cite{zhang2009gradient}

	\subsubsection{Particle Swarm Optimization}
		In Particle Swarm Optimization (PSO) a number of particles is randomly distributed over an unkown space of a problem or function. 
		Each particle - in our case a robot - evaluates its current location according to a certain fitness function.
		Then it calculates the best position to go to according to its own historical best position and the historical best positions of the particle(s) in its neighbourhood.
		To prevent the particles from agglomeration a certain randomness is often implemented. 
		By continuously looking for a better position by helping each other, the swarm of particles eventually positions itself at the position of source. \cite{poli2007particle,pugh2007inspiring,derr2009multi}\\
		\\
		In a paper by Jim Pugh and Alcherio Martinoli an attempt to model robotic swarm search based on PSO has been made. \cite{pugh2007inspiring}
		To achieve this a couple of modifications had to be made, for example: changing the normally discrete time to continuous, handling the movement limitations and collisions of robots and limiting the particle neighbourhood of each robot, which is often unlimited in general PSO.
		In this particular paper the algorithm is \emph{location-based} and \emph{range-based}.
		Attempts to make the algorithm \emph{location-free} have been made.
		Conclusions are that either location-based or location-free the performance improves when upscaling the amount of robots.
		When testing the location-free variant, the increase in performance is very abrupt at a certain number of robots while in the location-based variant the performance improves gradually.
		Furthermore detection of the source with small communication range achieves poor results, but improves dramatically as the range increases.\\
		\\
		Another implementation of a decentralized application of PSO is implemented in a paper by Derr. \cite{derr2009multi}
		The developed algorithm is completely \emph{location-free} and \emph{range-based}.
		The targets are equipped with a cell phone that radiates a radio frequency signal that can be detected by the robot, which can wirelessly communicate with limited range.
		The paper shows that a distributed algorithm based on PSO can easily overshoot sources
		This can be prevented with a correction on the standard PSO formula.
		Furthermore it concludes by experiments that the variation in received signal strenghts (RSS) in an indoor environment significantly increases the robot search time in finding the sources.

	\subsubsection{Glowworm Swarm Optimization}
		In Glowworm Swarm Optimization (GSO) the idea is to distribute "glowworms" randomly over the area.
		The glowworms, according to the fitness function, carry a certain lumeniscence quantity called luciferin. 
		The closer they get to the source the more luciferin they contain and the more they attract other glowworms. 
		At every movement step each glowworm moves towards a neighbour within a certain range that carries more luciferin, so they eventually conglomerate at the source(s). 
		The glowworms have a communication range that varies at each step with a certain randomness, to make sure not all robots focus on the same source. \cite{krishnanand2005detection}\\
		\\
		In a paper by Krishnanand a GSO based distributed robotic algorithm is succesfully implemented. \cite{krishnanand2005detection}
		In comparison to PSO based algorithms this algorithm is completely \emph{memoryless}.
		The number of sources found by this particular algorithm is proven to be a strong function of the sensor range.
		Since the number of sources to find is not known in advance, the sensor range is made a varying paramter.
		Besides it shows that when blocking regions the inter-robot communication helps to still let robots select a feasible direction towards the source.

	\subsubsection{Biasing Expansion Swarm Approach}
		In the Biasing Expansion Swarm Approach (BESA) robots have communication possibilities over limited range and together create a ad-hoc wireless network for global communication capibility.
		The algorithm is therefore \emph{location-free} and \emph{range-based}.
		Each robot maintains an occupancy grid map to represent the environment wchich is initiated with all cells unexplored.
		After deployment robots share their locations and sensed concentrations with the swarm.
		To uphold seperation a robot does not enter a cell which is already occupied and maintains cohesion by using a gradual expansion algorithm.
		Robotos can therefore only move to cells that are unexplored, unoccupied and next to another robot in the swarm: the expansion cells.
		To make sure the swarm moves to the emission source, each expansion cell is given a certain biasing parameter, based on the number of robots, the distance between the expansion cell and the robot and the concentration sensed at the particular cell.
		The robot will choose the expansion cell with the highest parameter, so that the swarm will eventually move to the direction of the higher concentrations.
		The BESA algorithm developed in this paper performs twice times better than the general gradient-based approaches. \cite{cui2004swarm}

	\subsubsection{Biased Random Walk}
		A rather simple algorithm inspired by nature is the Biased Random Walk (BSR) approach inspired by bacteria as is described in a paper by Dhariwal \cite{dhariwal2004bacterium}.
		Implementations of this algorithm do not use communication or any form of localization and are therefore \emph{location-free} and neither range-based or range-free.
		Robots only have the possibility to perform two actions: move or tumble. 
		When tumbling the robot just turns into a new random direction. 
		When moving the robot travels a certain distance into the chosen direction. 
		If the robot senses some form of emission from the source, the tumble frequency is lowered, so the distance moved per step increases.
		Eventually the robots will conglomerate at multiple sources.

	\begin{table}[H]
  \renewcommand{\arraystretch}{1.3}
  \caption{Overview of Common Localization Algorithms}
  \label{table_alg_localization}
  \centering
    \begin{tabular}{|l|l|l|l|l|l|}
    \hline
    \bfseries Algorithm & \bfseries Range-type & \bfseries Location-type & \bfseries Performance & \bfseries Scalability\\
    \hline
    \bfseries PSO & Range-based & Location-free & High & High\\
    \hline
    \bfseries GSO & Range-based & Location-free & High & High\\
    \hline
    \bfseries BESA & Range-based & Location-free & Medium & Medium\\
    \hline
    \bfseries Gradient-based & Range-free & Location-based & Medium & Low\\
    \hline
    \bfseries BRW & - & Location-free & Low & High\\
    \hline
    \end{tabular}
  \end{table}

	\subsubsection{Problems}
	The main problem of localization using swarms in the \emph{range-based} approaches is the way robots should cooperate and learn from each other. 
	In other words: how big should the neighbourhood be in which robots cooperate to find sources? 
	One might think it is ideal to use information of all robots.
	However, in case of multiple sources, the best position of all robots can be misleading and cause premature convergence. 
	Therefore in PSO a particle neighbourhood within a certain range or a constant maximum value of neighbours is often chosen. 
	In GSO a range varying each step is implemented to make sure multiple sources can be found.
	Finally in BESA based approaches the robots create an ad-hoc network to create global communication. 

	\subsubsection{Remaining problems}
	There are a lot of factors that affect the performance of localization of multiple sources.
	For example the complexity of the implemented algorithms, the number of robots used or the number of sources that is being searched for. 
	As is mentioned in a paper by McGill however there is a lack of validation cases and reference algorithms that form a ground truth for comparative analysis. \cite{mcgill2011robot} 
	Therefore it is relatively difficult to compare the developed algorithms for (multi-)source localization with robotic swarms. 
	Another question in this field is which approach should be used after a source has been found. 
	Some papers propose to collect the source/target and search further, but this is not always possible.

\section{Collective Transport}
  \label{sec:CollectiveTransport}
  %!TEX root = ../Bachelorseminar-RoboticSwarms.tex

Collective transport of objects is the problem in which a swarm of robots locates a payload and collectively moves the payload to another place, like a home base. 
This problem is a composite problem, consisting of the already mentioned problems formation and source localization. 
Still, we want to discuss this problem because it is an important problem in the research field of robotic swarms.
%This can be compared to the path-planning problem; although this implies that a path is made to a certain place, which is not the case in the collective transport problem. \cite{hoff2010two}  
%The localization problem is an important part of the collective transport problem and has much overlap.\\
Transporting objects by robotic swarms has many potential applications in many settings, from agriculture to construction to disaster relief. 
Especially in dangerous settings like war zones or radio-active areas, robotic swarms can be a powerful tool to safely retrieve many objects. 
For example, recently Amazon, a large online retailer, announced it would make use of unmanned flying robots to deliver parcels at everyone's door. \footnote{Amazon's PrimeAir service, to be released sometime in 2015. www.amazon.com/b?node=8037720011}

\subsection{Algorithms}

Collective transport can be separated into two subproblems: providing safe and reliable transport and moving a payload to its destination.
These respectively correspond to the formation problem and the source localization problem.

%If a homogeneous swarm is controlled by a single ground station, the algorithm is not considered with finding a payload.
%In algorithms in which swarms are completely autonomous, the focus is on finding the payload and transporting it to its destination.
%Instead it is focused on safe, accurate and scalable transport. 
%So, we can categorize these subproblems as user-controlled transportation problems and autonomous transportation problems.
%In user-controlled transportation, the location the swarm operates in is known, so the algorithms used are location-based.
%In autonomous transportation, the location is not known and the algorithms are location-free.

When considering the formation problem, the algorithm is not considered with finding a payload. 
Then, we can conclude that the swarm in this algorithm is not completely autonomous and that the whole swarm is controlled by a single controller.
This implies that algorithms used for the formation problem are location-based, because the location is known by the controller. \cite{Michael2011cooperative,mas2012object}
When considering the source localization problem, the algorithm is not concerned with transporting the payload. 
Then, we can conclude the swarm in this algorithm works autonomously. 
This implies that the algorithms used for the source localization problem are location-free, because the location is now known by the controller. \cite{Rubenstein2013collecive, sugawara2012object}

\subsubsection{Formation transport}
%A robot, especially a flying robot, can only carry a certain weight, which poses a problem. 
%But with a robotic swarm of robots it is possible to transport larger weights in a scalable system, adding swarm robots until you can carry the desired weight. 
%So, a swarm of robots is useful for transporting large and/or heavy objects. 
There are two different properties important for formation transport: performance and scalability. 
We define \emph{performance} as in how well the robots can stay in formation in dynamic environments. \\
%To ensure safe and stable transport, swarm robots have to be coordinated in such a w ay that the payload can be manipulated to a desired position and orientation.

The first algorithm we discuss is the \emph{Aerial Equilibrium} algorithm. \cite{Michael2011cooperative}
This algorithm for aerial transport is implemented by attaching cables from quadrocopters to the payload.
Then, each robot calculates its own movement in the swarm by a mathematical model. 
Because the swarm robots do not use local communication but get instructions from a user, this algorithm can be defined as a range-free algorithm. 
A change in the position of a robot is quickly communicated through the mathematical model, resulting in a good performance.
Nevertheless, the scalability is not that good, because adding multiple robots will quickly result in robots getting in each others way.
This is because the chance of touching each others rope increases with every robot added. \\

%An algorithm for such transport is given in the paper Cooperative manipulation and transportation with aerial robots.
%This paper formulates general conditions for the payload equilibrium but can be scaled to many swarm robots. 
%The mathematical model that is defined in the paper is tested with three quadrocopters and offers a stable transportation for objects, with a degree of freedom related to the amount of robots used and can thus be implemented in the field.

The second algorithm we discuss is called \emph{Cluster Space Control} algorithm. \cite{mas2012object}
It uses a multi-robot formation control framework called cluster space control.
It is utilized to control a swarm of four four-wheeled robots.
A user using this type of transportation uses a joystick to input user controls to the whole swarm, after which the centralized control framework specifies the formation and the position and shape of the swarm.
Because of the centralized communication there is no local ranged communication, so this algorithm is range-free. 
This control framework allows the user to effectively transport large objects with a scalable set of robots. 
The performance for this algorithm is very high, because the framework responds well to changes in the environment. 
The scalability is low, because the framework has to hold connections to every robot. \\

%The algorithm can be used for transporting large objects on the ground, in the water and in the air.
%necessary to coordinate the motions of the robots in the group. 
%This is because control approach used is called cluster space control, and is utilized to control swarm of four four-wheeled robots. 

These two algorithms solve the same problem, namely the problem of transporting large and/or heavy objects, in two entirely different ways. 
They are both range-free and location-based, but differ mostly in scalability. 
As both algorithms have centralized communication, scalability is low for both.
We summarize the properties in the Table~\ref{table_alg_coltrans1}.

  \begin{table}[H]
  \renewcommand{\arraystretch}{1.3}
  \centering
\scalebox{0.85}{
  \begin{tabular}{|l|l|l|l|l|}
    \hline
    \bfseries Algorithm & \bfseries Range &\bfseries Location &  \bfseries Performance & \bfseries Scalability\\
    \hline
     Aerial Equilibrium & Range-free & Location-based & Medium & Low\\\hline
     Cluster Space Control & Range-free & Location-based & High & Low\\\hline
    \end{tabular}
    }
  \caption{User-controlled swarm transportation algorithms}
  \label{table_alg_coltrans1}
  \end{table}

In the first algorithm, each swarm robot individually calculates its position to the other robots and the payload and the resulting tension on the payload. 
This way, the rotation and position can be controlled by the position and tension of each robot. 
With the second algorithm, that uses the cluster \emph{cluster space control}, the calculations are not done individually. 
Instead, the framework calculates the position of each robot and the orientation of the payload. 
Therefore these methods both provide stability to the transported payload but in two completely different ways. \\

The problems these algorithms have are that the robots should be coordinated to rely on the location of other robots.
With the mathematical model these are calculated through the tension of the cable attached to each robot, but in the framework it is calculated centrally. 
Problems that are not treated in these articles is how these swarms can locate a payload and autonomously can deliver it to a location. 
Another problem for which no solution is provided is that when these swarms grow to account for heavier objects, the robots will get in each other's way. 
Especially considering transportation, a technique in which every robot must latch on to a payload, this can be hard to practically scale. 

\subsubsection{Source localization transport}
The source localization problem consists of finding the payload and transporting it to its destination.
This subproblem can again be divided in four smaller problems: finding the payload, grabbing the payload, finding target location, moving the payload to the location.
By defining these subproblems, the problem is easier to solve and easier to explain each algorithm. 
The \emph{performance} is defined in how fast the payload is moved to the destination. \\

The first algorithm we discuss is the flocking algorithm. \cite{Rubenstein2013collecive}
Simply put, it locates the payload through a flocking algorithm and let the swarm robots push the payload to the goal location. 
First every robot looks if it can see the payload location. If so, move towards the location. 
If not, the robot compares its own heading with the other robots nearby and corrects it until the robot finds the payload location.
In this paper, the payload location and the goal location are found with light sensors, communicates headings with infrared sensors and uses bump sensors to avoid obstacles. 
The robot latches on to the payload with a velcro strip.
After finding the payload, the robot pushes from an arbitrary side, calculating the amount of force needed in relation to the middle point of mass with its bump sensors and transporting it to the goal location with its light sensor.  
And because the robots latch onto an arbitrary side, scaling of the swarm is extremely effective. \\ 

%Each robot uses one of two behaviors.
% A robot that can see the goals aligns itself towards the goal while moving forward.
% A robot that cannot see the goal uses its IR sensors to determine the heading difference between itself and its neighbors.

%Another widely used algorithm for transporting objects is the \emph{pheromone} algorithm. 
%This algorithms relies on leaving a trail of marks that help guide other swarm robots. 
%Although this is algorithm can be used for foraging, it can also be used for retrieving a single payload leading many robots to transport a single payload.
%This algorithm is used in the paper Cooperative Transportation by Swarm Robots Using Pheromone Communication. \cite{fujisawa2013cooperative}
%In this paper, this algorithm is tested with swarm-bots which can release and detect spots of ethanol left on the ground, indicating a spot they have to follow. Although the pheromone algorithm implies that pheromones should be used, different varieties of markings can be used, including wireless sensor beacons, drops of paint and even objects. This algorithm is particularly useful for guiding many robots to a payload, if for example the payload is very heavy or if many robots are needed for accurate and/or stable transportation. This algorithm is range-based; it calculates the exact distance to a trail. This algorithm is difficult to implement in dynamic environments, as made marks can dissapear. \\

The following algorithm uses very simple robots, which is desirable. \cite{sugawara2012object}
It is based on granular convection, also known as the Brazil Nut Effect. 
The algorithm can be implemented in three different ways: (1) A swarm with homogeneous robots with no explicit communication, 
(2) a swarm of heterogeneous robots composed of robots that change their direction with two different probabilities, 
and (3) a swarm of heterogeneous robots that uses local communication to adjust the fraction of robots that change their direction with higher probability. 

Each implementation has a higher performance than the last one, but also increase in communication. \\ 
The way this algorithm works is as follows. 
The goal of the swarm robots is to transport the object to the goal location. 
Each swarm robot vibrates with a random force. The goal location outputs a repulsive force. 
As the robots are randomly placed around the payload and the goal location, the payload gets kicked around by the vibrating swarm robots.
This continues until the payload is kicked to the target location.
When the swarm increases, the amount of time it costs to transport the payload significantly decreases. 
With heterogeneous robots, there is more communication between robots to organize the vibrations of the robots. \\
%With heterogeneous robots, there is more communication between robots to lower the density of the robots around the payload, because slows the process of moving the payload.\todo{SB: I don't get this.}

The algorithms we describe are all location-free, as mentioned earlier. 
Flocking is a range-based algorithm, because the robots communicate the headings locally. 
Of course, granular convection with local communication is also range-based by definition. 
Although granular convection is mostly range-free, its scalability overall is high, because there is no central communication. 
With the granular convection method, adding local communication and using heterogeneous robots increases the performance.
This is also not detrimental to the scalability of the approach. 

  \begin{table}[H]
  \renewcommand{\arraystretch}{1.3}
  \centering
\scalebox{0.85}{
  \begin{tabular}{|p{7cm}|l|l|l|l|}
    \hline
    \bfseries Algorithm & \bfseries Range & \bfseries Location & \bfseries Performance & \bfseries Scalability\\
    \hline
    Flocking & Range-based & Location-Free & High & High\\\hline
    %Pheromone & Range-based & Location-Free & Medium & Medium\\\hline
    Homogeneous granular convection & Range-free& Location-Free & Low  & High\\\hline
    Heterogeneous granular convection & Range-free& Location-Free & Medium & High\\\hline
    Heterogeneous granular convection with \newline local communication & Range-based& Location-Free & High & High\\\hline

    \end{tabular}
    }
    \caption{Overview of Collective Transport Algorithms}
  \label{table_alg_coltrans2}
  \end{table}


%A problem these algorithms ran into was how to locate the payload, how to move the payload and how to find and reach the goal location. 
%The main problem is locating the payload, something which the granular convection algorithm actually never does.  
%What these three algorithms have in common is that the transportation process speeds up significantly when adding more swarm robots, but reaches a cap when it becomes ineffective due to maximal density. 
%For this reason, small, easy and cheap to produce swarm robots are required. 
%The granular convection algorithm has the cheapest robots, but is not necessarily the best.
%This is because when the robots become slightly larger and more complicated, more sophisticated and faster algorithms could work.\todo{SB: Whole alinea is formulated a bit vague and is repetitive compared to previous section, take a look at it.} \\
%These three algorithms are very different from each other, but they all serve a common goal. 

In the mentioned algorithms, moving the payload is mostly done by pushing. 
In real-life applications this would not always be a good option and a more sophisticated method of attaching should be used. 
When considering search-and-rescue operations for example, a solution to effectively grab the person to be rescued should be thought of. 
This is a problem that is yet to be identified in these algorithms and poses a good question for the future.

  
% \section{Path-planning}
%   \label{sec:Path-planning}
%   %!TEX root = ../../Bachelorseminar-RoboticSwarms.tex

Path-planning is the problem of a robotic swarm for each robot to find the optimal collistion-free path from the start location to the target location. \cite{qin2004path}.
Another term for path-planning is path-finding. 
This is a problem found in many robotic swarms applications.
This is because when robots move to a location in any distributed algorithm, this should be done as fast as possible. 
Path finding can be used in multiple real-life applications.
These applications include trying to find the optimal path to food sources and bringing food back to the base, which is often called foraging \cite{hoff2010two}.
Another application is finding an optimal path to survivors from a disaster, for search-and-rescue operations.
Of course, this can also be extended to path-planning for oil spills, crime victims or other targets.\cite{pugh2007inspiring} 
Path-planning algorithms are often combined with some form of particle swarm optimization. \cite{poli2007particle}. \\

There are many algorithms that can be used to solve the path-planning problem. 
We will point out a few of the most important and recent algorithms. \cite{hoff2010two} \cite{bhattacharjee2011multi} \cite{jung2010multi} \cite{kala2012multi}
We will discuss these algorithms, pointing out how they work and what problems are solved. \\

The first algorithm is the foraging algorithm. \cite{hoff2010two}
Foraging algorithms are based on the foraging behavior of ants to find food sources. 
Ants can leave chemical pheromones behind for other ants, which they use to decide which way they go. 
Two different algorithms can be used for this: the Virtual Pheromone algorithm and the Cardinality algorithm.
The Virtual Pheromone algorithm uses two different types marks instead of many, to keep the robots low-level. 
For the same reason, the marks are made virtual by direct local communication between robots in the swarm.
The core of the algorithm is that some robots will stop their search for the target food source, and become 'pheromone robots', also known as beacons.
These beacons send out marks as floating point numbersfor other foraging swarm robots, telling them where or where not to move. \\
The second foraging algorithm is the Cardinality algorithm. 
This works similar to the Virtual Pheromone algorithm, as in that walker robots can become beacons. 
The difference is that the beacons send out integer values, indicating how many other beacons it would hear, including the nest beacon. 
Every integer value of beacon can then be interpreted as the number of beacons between that beacon and the nest. \\
Conclusive results are produced in the article that the Cardinality-algorithm outperforms the Virtual Pheromone algorithm, proving that the ant pheromone algorithm can be improved.
These two algorithms both use only local communication and every robot does not detect exact range between other robots.
So, the two algorithms are bot location-free and range-free. \\

The second algorithm we will discuss is the Artificial Bee Colony algorithm. \cite{bhattacharjee2011multi}
Again, this algorithm is inspired by nature, but this time is based on the stoach behavior of foraging in bees.
In this algorithm, the robots (from this point on referenced as bees), can be divided in three groups:
\begin{itemize}
	\item A bee waiting on a dance area for making decision to choose a food source is called an onlooker.
	\item A bee going to the food source visited by it previously is named as employed bee.
	\item A bee carrying out random search is called a scout.
\end{itemize}
The algorithm then consists of four phases:
\begin{enumerate}
	\item The behaviors are randomly distributed among all the robots (according to a certain distribution).
	\item The employed bees move to find a new or better food source, if they can find it based on visual information. 
	\item The onlooker bees then calculate the best places to gather food depending on the input from the employed bees.
	\item Finally, if there are food sources of which the path to it can not be improved and have reached a limit, they are abandoned and replaced by scouts.
\end{enumerate}

Steps 2, 3, and 4 are continuously repeated, until an optimal path is created for all the employed bees.
This algorithm does make use of a given environment and the path-planning is done with local sensors. 
So, this algorithm is location-based and range-based.\\

A third algorithm we will research is the Multihop Communication algorithm. \cite{jung2010multi}
This algorithm uses robots that communicate via wireless communication, utilizing multihop communications to use in networked robotics. 
The problem that is defined here is how a swarm of robots can find its way out of a random maze as fast as possible.
Each robot is equipped with sensors that detec walls and paths, and can save the learned map in its memory and move toward its desired direction. 
Each robot then, while exploring the maze, update his learned map with the newly explored territory, after which he relays this to the next robot in the multihop routing sequence. 
This way, every robot eventually shares the same map and can extrapolate the fastest path out of the maze. 
A major drawback of multihop communications is the multi-hop delay which is generated. 
This delay is decreased for the network by letting each robot send its map to a robot furthest away. 
Another problem is that the walls of the maze limit the communication range of the wireless network.
This algorithm lets robots keep track of its own location,  and needs the exact range of all the swarm robots to determine the robot furthest away.
So, this algorithm is location-based and range-based. \\

The last algorithm we will discuss here is Genetic Programming. \cite{kala2012multi}
The paper presents a co-evolutionary genetic programming based planning of multiple robots. 
Like the previous algorithm, the goal of this swarm is to find the optimal path in a random given maze.
Each robot has its own optimization process that is based on the principles of grammatical evolution. 
All the robots try to optimize the path of the individual robots. 
But, inter-robot collisions get in the way of this optimization. 
These collisions are then induced by co-operation of the robots. 
After computing these individual paths, a master genetic algorithm runs to slect the best paths of the robots out of all the computed individual paths. 
A major problem is the speed of the communication.
Every robot does calculate its own position in the maze through other robots, and calculates the range between robots for collisions.
So, this algorithm is location-based and range-based. \\

We have summarized these algorithms in the following table.

  \begin{table}[H]
  \renewcommand{\arraystretch}{1.3}
  \caption{Path-planning algorithms}
  \label{table_alg_coltrans1}
  \centering
  \begin{tabular}{|p{3cm}|l|l|l|l|}
    \hline
    \bfseries Algorithm & \bfseries Range &\bfseries Location &  \bfseries Performance & \bfseries Scalability\\
    \hline
     Virtual Pheromone & Range-free & Location-free & Medium & High\\\hline
     Cardinality & Range-free & Location-free& High & High\\\hline
     Artifical Bee Colony & Range-based & Location-based & Medium & Medium\\\hline
     Multihop Communication & Range-based & Location-based & High & Low\\\hline
     Genetic Programming & Range-based & Location-based & High & Low\\\hline
    \end{tabular}
  \end{table}

As can be concluded from the table, we can see that the nature-inspired algorithms are more scalable then the other algorithms
This makes sense, as the algorithms used by ants and bees are used in settings where hundreds of insects work together. 
But, as stated in the descriptions, algorithms like Multihop Communication and Genetic Programming try to map the whole environment. 
This results in a higher accuracy for finding the optimal path, even in very complex environments. 
The downside for this is a lower scalability. 
Future discussion is how this scalability for these algorithms can be made better, while still keeping the accuracy and performance. \\

 
\section{Discussion}
  \label{sec:Discussion}
  %!TEX root = ../Bachelorseminar-RoboticSwarms.tex

NOTES

- scalability, global communication, 
- lots of overlap

WHAT WE MISS
- mapping
- foraging
- target localization?
- robot localization

In the past sections, we reviewed the most common problems found in the field of robotic swarms. 
These problems however, do overlap, because the problems found in this field often consist of multiple different problems. 
We focused on each problem, highlighting the communication methods of every solution and properties of these communication methods. 
These properties can be summarized in a Venn diagram, allowing for a compact overview of these solutions. 

	\begin{figure}[!ht]
		\label{venn_diagram}
		\caption{Venn-diagram of all mentioned approaches}

		\centering

		\def\loclb{(180:2.0cm) circle (2.0cm)}
	  	\def\loclf{(0:2.0cm) circle (2.0cm)}
	  	\def\locrb{(90:2.0cm) circle (2.0cm)}
	  	\def\locrf{(270:2.0cm) circle (2.0cm)}

	    \begin{tikzpicture}
			% standard figures
			\draw \loclb node [text=black] {Location-based};
			\draw \loclf node [text=black] {Location-free};
			\draw \locrb node [text=black] {Range-based};
			\draw \locrf node [text=black] {Range-free};

			% Range-based location-free
			\draw[dashed,-] (1,1) -- (2,5.5) node[anchor=north west] {};
			\node[draw,align=left,anchor=west] at (2,5.5) {
				Leader-follower \ref{sec:Formation}\\
				Virtual structure \ref{sec:Formation}\\
				Virtual space \ref{sec:Formation}\\
				Teamwork control \ref{sec:Formation}
				Particle Swarm Optimization \ref{sec:Localization}\\
				Glowworm Swarm Optimization \ref{sec:Localization}\\
				Pheromone \ref{sec:CollectiveTransport}\\
				Heterogeneous granular convection WLC \ref{sec:CollectiveTransport}\\
				Random Walk \ref{sec:Dispersion}\\
				Follow wall \ref{sec:Dispersion}\\
				Directed Dispersion \ref{sec:Dispersion}\\
				Seek open \ref{sec:Dispersion}\\
				Fiducial \ref{sec:Dispersion}\\
				Clique-intensity \ref{sec:Dispersion}
			};

			% Range-free, location-free
			\draw[dashed,-] (1,-1) -- (2,-4.5) node[anchor=north west] {};
			\node[draw,align=left,anchor=west] at (2,-4.5) {
				Behavior-based \ref{sec:Formation}
				Fuzzy control \ref{sec:Formation}\\
				Flocking \ref{sec:CollectiveTransport}\\
				Homogeneous granular convection \ref{sec:CollectiveTransport}\\
				Heterogeneous granular convection \ref{sec:CollectiveTransport}\\
				Aerial Equilibrium \ref{sec:CollectiveTransport}\\
				Virtual Pheromone \ref{sec:Path-planning}\\
				Cardinality \ref{sec:Path-planning}
			};

			% location-free
			\draw[dashed,-] (3.5,0) -- (5,0) node[anchor=north west] {};
			\node[draw,align=left,anchor=west] at (5,0) {
				Biased Random Walk \ref{sec:Localization}
			};

			% location-based
			\draw[dashed,-] (-3.5,0) -- (-5,0) node[anchor=north west] {};
			\node[draw,align=left,anchor=east] at (-5,0) {
				Frontier-based \ref{sec:Exploration}\\
				Gradient-based \ref{sec:Localization}
			};

			% Range-free, location-based
			\draw[dashed,-] (-1,-1) -- (-1.5,-4.5) node[anchor=north west] {};
			\node[draw,align=left,anchor=east] at (-1.5,-4.5) {
				Biasing Expansion Swarm Approach \ref{sec:Localization}
			};

			% Range-based, location-based
			\draw[dashed,-] (-1,1) -- (-1.5,4.5) node[anchor=north west] {};
			\node[draw,align=left,anchor=east] at (-1.5,4.5) {
				Frontier-based \ref{sec:Exploration}\\
				Market Economy based \ref{sec:Exploration}\\
				Cluster Space Control \ref{sec:CollectiveTransport}\\
				DFLF \ref{sec:Dispersion}\\
				BFLF \ref{sec:Dispersion}\\
				Artificial Bee Colony \ref{sec:Path-planning}\\
				Multihop Communication \ref{sec:Path-planning}\\
				Genetic Programming \ref{sec:Path-planning}
			};

			% Range-based
			%\draw[dashed,-] (0,3) -- (0,4.5) node[anchor=north west] {};
			%\node[draw,align=left,anchor=south] at (0,4.5) {

			%};

			% Range-free
			%\draw[dashed,-] (0,-3) -- (0,-4.5) node[anchor=north west] {};
			%\node[draw,align=left,anchor=north] at (0,-4.5) {
			
			%};
		\end{tikzpicture}
    \end{figure}

We see that most solutions are range-based and location-based. 
In practice, these solutions are not the most desirable solutions, because of two reasons. 
The first reason is scalability. 
Because when an algorithm is location-based and range-based, the robots used in the swarm have to be more advanced to correctly handle complex information. 
This also introduces a lot of overhead, decreasing communication speed when the robotic swarm increases in size. 
The second reason is that algorithms that are location-based can often not be used in dynamic locations, especially in algorithms where the operating environment has to be defined beforehand. 


\section*{Acknowledgment}
The authors would like to thank A. Loukas and A.S. Pruteanu for their valuable comments and suggestions to improve the quality of this survey.

\clearpage
% Default is plain
\bibliographystyle{plain}

% Usage of multiple bib input files through seperation by comma.
\bibliography{./bib/Bibliography}

\end{document}

